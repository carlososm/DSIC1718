\section{Presupuesto}\label{sec:presupuesto}
1310 euros

\subsection{Introducción}
\par A lo largo de este apartado se procederá a evaluar la estimación de costes que supondrá el desarrollo de este proyecto. Para ello, en primer lugar, se indicará el personal a cargo del proyecto, así como el coste por hora de trabajo de cada uno de ellos. Tras realizar una estimación de horas realizadas por cada uno de los empleados, se estimará el coste total del salario del personal que trabajará en el proyecto.
\par Así mismo, se hará una estimación del coste del material informático utilizado (tanto el referente al software como al hardware), del material fungible, del material del pruebas, de los viajes y dietas y de los costes indirectos.
\par Con todo ello, se proporcionará una estimación del coste total del proyecto, que será utilizada para el presupuesto del mismo y para el documento de oferta remitida.

\subsection{Cálculo de Costes}

\subsubsection{Resumen del personal a cargo}
\par Para el desarrollo del proyecto necesitaremos un total de siete miembros trabajando en el equipo. En este personal deberá haber un Jefe de Proyecto, encargado de liderar el equipo y ofrecer las directrices necesarias para el desarrollo del mismo. Así mismo, se contará con un Analista de Sistemas, un especialista en la Gestión de Configuración, un responsable de pruebas y otro de calidad y dos desarrolladores.
\par Así, en la tabla \ref{tab:personal} se puede observar qué empleados formaran parte del equipo de trabajo de este proyecto y cuál será su salario por hora de trabajo en función de su rol (y no de cada miembro del equipo).
\par Por otro lado, en la tabla \ref{tab:repHoras} se puede observar el número de horas realizado por cada miembro del equipo en cada una de las tareas.
\par Para esta estimación del coste por hora de cada una de las personas del proyecto (no de cada rol) se ha realizado la media ponderada del coste de cada uno de los roles de ese miembro del equipo.

\begin{table}[H]
\begin{center}
\begin{tabular}{l l l c}

\textbf{CARGO} & \textbf{NOMBRES} & \textbf{ROL} & \textbf{COSTE/HORA}\\ \hline \hline
Jefe de Proyecto & Jorge Hevia & JorgeHeviaJP & 30  \\
Analistas de Sistemas & Carlos Olivares & CarlosOlivaresAS & 25\\
Gestión de Configuración & Carlos Olivares & CarlosOlivaresCF & 25\\
Responsable de Calidad & Luis Cabrero & LuisCabreroCA & 25\\
Responsable de Pruebas & Luis Cabrero & LuisCabreroPR & 20\\
Desarrollador jefe & Jorge Hevia & JorgeHeviaDJ & 20\\
Desarrollador 1 & Carlos Olivares & CarlosOlivaresD1 & 15\\
Desarrollador 2 & Luis Cabrero & LuisCabreroD2 & 15\\ \hline \hline
\end{tabular}
\caption{Resumen de personal.}
\label{tab:personal}
\end{center}
\end{table}



\begin{center}
\begin{longtable}{lcccccc}

%HEAD
& \textbf{Documentación}	&	\textbf{Análisis}	&	\textbf{Diseño}	&	\textbf{Implementación}	&	\textbf{Instalación}	&	\textbf{TOTAL} \\
& \textbf{(horas)}	&	\textbf{(horas)}	&	\textbf{(horas)}	&	\textbf{(horas)}	&	\textbf{(horas)}	&	\textbf{(horas)} \\
\hline
\hline
\endfirsthead
& \textbf{Documentación}	&	\textbf{Análisis}	&	\textbf{Diseño}	&	\textbf{Implementación}	&	\textbf{Instalación}	&	\textbf{TOTAL} \\
& \textbf{(horas)}	&	\textbf{(horas)}	&	\textbf{(horas)}	&	\textbf{(horas)}	&	\textbf{(horas)}	&	\textbf{(horas)} \\
\hline
\hline
\endhead

%FOOT
\hline \multicolumn{7}{r}{\textit{Continúa en la siguiente página}} \\
\endfoot
\endlastfoot

%table
Jorge Hevia	&	16	&	0	&	3	&	7	&	0	&	26	\\
Luis Cabrero &	7	&	5	&	9	&	6	&	0	&	27	\\
Carlos Olivares &	35	&	1	&	7	&	4	&	2	&	49	\\	\hline \hline
\textbf{TOTAL}	&	\textbf{58}	&	\textbf{6}	&	\textbf{19}	&	\textbf{17}	&\textbf{2}	&	\textbf{102}	\\	\hline

\caption{Reparto de horas}\\
\label{tab:repHoras}
\end{longtable}
\end{center}




\subsubsection{Salarios de los empleados}
\par Tras estudiar en el apartado anterior el número de horas que se estima realizará cada uno de los empleados en el proyecto, y sabiendo también el coste de los mismos por hora, a continuación se expone el salario total que percibirá cada uno de los empleados. Es esta la información que contiene la tabla \ref{tab:costePersonal}.

\begin{table}[H]
\begin{center}
\begin{tabular}{l c c c}
\textbf{NOMBRE} & \textbf{TOTAL HORAS} & \textbf{COSTE (\euro/hora)} & \textbf{COSTE (\euro)} \\ \hline \hline
Jorge Hevia & 26 & 25 & 650,00\\
Luis Cabrero & 27 & 20 & 540,00\\
Carlos Olivares & 49 & 21,70 & 1.061,70\\ \hline \hline
\textbf{TOTAL} & & & \textbf{2.251,70} \\ \hline
\end{tabular}
\caption{Coste de empleados.}
\label{tab:costePersonal}
\end{center}
\end{table}



\subsubsection{Equipos informáticos}
\par Para el desarrollo del proyecto haremos uso de los equipos informáticos indicados en la tabla \ref{tab:hardware}. En ella se puede ver el coste total de los dispositivos. Sin embargo, al usarlos únicamente durante los cuatro meses que dura el proyecto, en la misma se indica el coste que supondrá el uso de los mismos durante ese periodo de tiempo (amortización). Para el cálculo de la misma, se ha supuesto que todos los equipos se amortizan en 3 años.

\begin{table}[H]
\begin{center}
\begin{tabular}{l c c c c }
\textbf{DESCRIPCIÓN} & \textbf{UNIDADES} & \textbf{PRECIO UNITARIO (\euro)} & \textbf{TOTAL (\euro)} & \textbf{AMORTIZACIÓN (\euro)}\\ \hline \hline
Ordenadores MAC & 1 & 1.400 & 1.400 & 466,70\\
Ordenadores HP & 2 & 800 & 1.600 & 533,30\\
Servidor AWS & 1 & 20\euro/mes & 80 & 80\\
Impresora & 1 & 400 & 400 & 133,30\\ \hline \hline
\textbf{TOTAL} & & & & \textbf{1.213,30}\\\hline
\end{tabular}
\caption{Hardware informático.}
\label{tab:hardware}
\end{center}
\end{table}



\subsubsection{Herramientas del software}
\par Serán necesarias las licencias de los programas indicados en la tabla \ref{tab:software} para el desarrollo del proyecto. Para la parte de desarrollo, se utilizará el control de versiones git mediante el programa BitBucket, y el editor de código Atom. Para el desarrollo de la documentación necesaria se utilizará Office. Para la gestión del proyecto y el control de tareas y tiempos se utilizarán los programas Toggle y Trello.
\par Así mismo, para la gestión general del proyecto y la comunicación entre miembros del equipo se usaran las \textit{suits} Google Apps for Work y Slack.
\par Por otro lado, para la elgaboración de las presentaciones y el material gráfico se utilizará tanto Office como Photoshop respectivamente.


\begin{table}[H]
\begin{center}
\begin{tabular}{l c c c}
\textbf{DESCRIPCIÓN} & \textbf{UNIDADES} & \textbf{PRECIO UNITARIO} & \textbf{TOTAL (\euro)}\\ \hline \hline
Licencias Office365 & 7 & 8,80\euro/mes & 52,8\\
Licencia Toggle & 7 & 9\euro/mes & 54\\
Licencia Trello & 7 & 10\euro/mes & 60\\
Licencia Slack & 7 & 7,5\euro/mes & 45\\
Google Apps for Work & 7 & 4\euro/mes & 24\\
Licencia Photoshop & 3 & 19,99\euro/mes & 119,94\\
Licencia Atom & 7 & 0\euro/mes & 0\\
Licencia BitBucket & 7 & 0\euro/mes & 0\\ \hline \hline
\textbf{TOTAL} & & & \textbf{355,74}\\ \hline
\end{tabular}
\caption{Software informático.}
\label{tab:software}
\end{center}
\end{table}



\subsubsection{Material fungible}
\par Será necesario distinto material de oficina, así como fotocopias y recambios de la impresora, para el desarrollo del proyecto. Pueden verse estos costes en la tabla \ref{tab:fungible}. Se estima que se imprimirán unas dos mil páginas entre los documentos internos, los presentados al cliente y los documentos oficiales requeridos. Sabiendo que el coste del tóner es de 42,95 \euro y estimando una duración de 1200 páginas por tóner, se requerirán dos tóners.
Como material de oficina, se necesitarán los folios usados (un paquete de 2500 tiene un valor de 24,36 \euro), bolígrafos (tanto normales como \textit{veleda}), grapadora con grapas y similar. Se estima el coste de todo ello en 200 \euro.


\begin{table}[H]
\begin{center}
\begin{tabular}{l c}
\textbf{DESCRIPCIÓN} & \textbf{TOTAL (\euro)}\\ \hline \hline
Recambios de impresora & 85,90\\
Material de oficina & 200\\ \hline \hline
\textbf{TOTAL} & \textbf{285,90}\\ \hline
\end{tabular}
\caption{Material fungible.}
\label{tab:fungible}
\end{center}
\end{table}


\subsubsection{Viajes y dietas}
\par A lo largo del proyecto se celebrarán reuniones con los distintos \textit{stakeholders} del proyecto, lo que collevará tanto gastos de la gasolina utilizada en los viajes como de las posibles comidas a las que serán invitados dichos \textit{stakeholders}. Así, se estima que se realizarán unos 5.000 km a lo largo del proyecto. Con un consumo medio de $5,7 litros /100km$ y un coste medio de gasolina de $1,41 \euro/litro$, el coste total de gasolina será de 400 \euro.

\begin{table}[H]
\begin{center}
\begin{tabular}{l c}
\textbf{DESCRIPCIÓN} & \textbf{TOTAL (\euro)}\\ \hline \hline
Gasolina & 400\\
\textbf{TOTAL} & \textbf{400}\\ \hline
\end{tabular}
\caption{Viajes y dietas.}
\label{tab:viajes}
\end{center}
\end{table}


\subsubsection{Costes indirectos}
En la siguiente tabla mostramos los costes indirectos derivados de las reuniones que mantendrá el equipo y su espacio de trabajo. Al no disponer de oficina física, no existen gastos de electricidad o similares. Sin embargo, si existen gastos asociados al alquiler de una sala co-working, que se alquilará durante 1 hora al día. Así se refleja en la tabla \ref{tab:indirectos}, cuyo cálculo se realiza mediante un coste de 40 euros la hora un total de 5 horas semanales durante las 15 semanas del prpyecto.

\begin{table}[H]
\begin{center}
\begin{tabular}{l c}
\textbf{DESCRIPCIÓN} & \textbf{TOTAL}\\ \hline \hline
Alquiler espacio co-working (Sala Tokio - Impact Hub Madrid) & 100\euro/semana\\ \hline \hline
\textbf{TOTAL} & \textbf{1.500\euro}\\ \hline
\end{tabular}
\caption{Costes indirectos.}
\label{tab:indirectos}
\end{center}
\end{table}


\subsection{Costes totales}
\par A continuación, se muestra el presupuesto final del proyecto, desglosando en los distintos costes que lo forman. La duración de dicho proyecto es de 21 semanas. El IVA aplicado es del 21\%.

\begin{table}[H]
\begin{center}
\begin{tabular}{l c}
\textbf{DESCRIPCIÓN} & \textbf{TOTAL}\\ \hline \hline
Sueldo del equipo de trabajo & 2.251,70\\
Amortización de Equipos informáticos & 1.213,30\\
Software informático & 355,74\\
Material fungible & 285,90\\
Viajes y dietas & 400\\
Costes indirectos & 1.500\\ \hline \hline
\textbf{TOTAL} & \textbf{6.024,64}\\ \hline
\end{tabular}
\caption{Resúmen de costes totales.}
\label{tab:resumenTotal}
\end{center}
\end{table}

En esta tabla se muestra el coste del proyecto sin I.V.A, así como, el riesgo y el beneficio a obtener por la empresa.
\begin{table}[H]
\begin{center}
\begin{tabular}{l c}
\textbf{DESCRIPCIÓN} & \textbf{TOTAL}\\ \hline \hline
Coste del proyecto (sin IVA) &  6.024,64\\
Riesgo (en porcentaje) & 15\% \\
Beneficio (en porcentaje)** & 15\% \\ \hline \hline
\textbf{TOTAL (sin IVA)} & \textbf{7.967,59}\\ \hline \hline
IVA 21\% & 1.673,19 \\\hline \hline
\textbf{TOTAL} &  \textbf{9.640,78}\\ \hline
\end{tabular}
\caption{Riesgos y beneficios.}
\label{tab:total}
\end{center}
\end{table}
