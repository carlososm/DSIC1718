\section{Presupuesto}

\par A continuación, se muestra el presupuesto final del proyecto, desglosando en los distintos costes que lo forman. La duración de dicho proyecto es de 21 semanas. El IVA aplicado es del 21\%. Para el cálculo de este presupuesto se han usado los costes calculados en el Capítulo \ref{chap:costes}.



\begin{table}[H]
\begin{center}
\begin{tabular}{l c}
\textbf{DESCRIPCIÓN} & \textbf{TOTAL}\\ \hline \hline
Sueldo del equipo de trabajo & 134.951,55\\
Amortización de Equipos informáticos & 1.432,93\\
Software informático & 355,74\\
Material fungible & 285,90\\
Material de pruebas & 617,72\\
Viajes y dietas & 3.400\\
Costes indirectos & 21.000\\ \hline \hline
\textbf{TOTAL} & 162.043,84\\ \hline
\end{tabular}
\caption{Resúmen de costes totales.}
\label{tab:resumenTotal}
\end{center}
\end{table}

En esta tabla se muestra el coste del proyecto sin I.V.A, así como, el riesgo y el beneficio a obtener por la empresa.
\begin{table}[H]
\begin{center}
\begin{tabular}{l c}
\textbf{DESCRIPCIÓN} & \textbf{TOTAL}\\ \hline \hline
Coste del proyecto (sin IVA) &  162.043,84\\
Riesgo (en porcentaje) & 15\% \\
Beneficio (en porcentaje)** & 15\% \\ \hline \hline
\textbf{TOTAL (sin IVA)} & 214.302,97\\ \hline \hline
IVA 21\% & 45.003,62 \\\hline \hline
\textbf{TOTAL} & 259.306,59\\ \hline
\end{tabular}
\caption{Riesgos y beneficios.}
\label{tab:total}
\end{center}
\end{table}



\par El pago del proyecto se repartirá de la siguiente forma:
\begin{itemize}[-]
\item Se realizará un primer pago del \textbf{30}\% (77.791,97)al realizar la firma del contrato.
\item Se realizará un pago del \textbf{50}\% (129.653,29)tras realizar la fase de diseño.
\item Se realizará un pago del \textbf{20}\% (51.861,31) cuando finalice el proyecto.
\end{itemize}


\par Este documento de oferta tiene validez hasta el 31/6/2017.


\signL{Alberto García Hernández}{Jefe de Proyecto}{./img/firma}
\vspace{-1.8cm}
\signR{CARSAFETY}{Cliente del proyecto}{}
