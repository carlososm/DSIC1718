\section{Retrasos}

\par En la siguiente tabla se muestran los retrasos leves que se han producido durante las dos últimas semanas, pues tras el reparto de tareas inicial a principios de semana, la fecha de entrega convenida por el director de proyecto es el jueves y muchas de las actividades no quedan terminadas hasta el fin de semana.
\par Es importante reaccionar ante esta situación pues la fecha de entrega límite de las tareas a realizar cada semana es el domingo a las 23:59 horas.
\par Hemos sufrido un retraso con el documento EVS de 2 días, pero afortunadamente no ha afectado a nuestra planificación, ya que los trabajadores han conseguido finalizar esa entrega sin que las entregas posteriores sufrieran ningún retraso.

\begin{table}[h]
\begin{center}
\begin{tabular}{ p{5cm} p{5cm} p{5cm} }
\hline
	Retraso  & Motivo & Acción a tomar \\ \hline
	EVS & Excesiva carga de trabajo & Mejor reparto de esta tarea \\ \hline
	IQS1 &  El miembro del equipo implicado está bastante ocupado con otros proyectos & Contactar con la persona implicada y preguntarle si puede terminar a tiempo \\ \hline
	GConf & El miembro del equipo implicado está bastante ocupado con otros proyectos  & Contactar con la persona implicada y preguntarle si puede terminar a tiempo \\ \hline
	Corrección documento de oferta (OFE) & Ningún recurso asignado  & Asignar esta tarea a alguien \\ \hline
	Corrección de la gestión de configuración (GCONF) &  Ningún recurso asignado & Asignar esta tarea a alguien \\ \hline
	Corrección de la gestión de calidad (PGCal) & Ningún recurso asignado  & Asignar esta tarea a alguien \\ \hline
\end{tabular}
\caption{Retrasos en el proyecto}
\label{tab:Retrasos en el proyecto2}
\end{center}
\end{table}
