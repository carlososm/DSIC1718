

\section{Establecimiento del Plan de Gestión de Riesgos}
\subsection{Identificación de Riesgos}
\subsubsection{Determinación de los orígenes y Categorías de los Riesgos}
\par Determinar el orígen de los riesgos es una tarea muy importante para definir el plan de calidad del proyecto, ya que para poder identificar correctamente los riesgos que pueden afectar al correcto funcionamiento del software es necesario saber su origen, y así poder redactar un plan para actuar en caso de que se produzcan.

\par Los riesgos que pueden afectar al proyecto pueden tener diferentes orígenes, tal y como se puede observar en la tabla \ref{tab:origenRiesgos}.






\begin{table}[h]
\begin{center}
\begin{tabular}{p{3,5cm} p{11cm}}
\textbf{Origen del Riesgo} & \textbf{Descripción}\\ \hline
Personal & Son aquellos provocados por el personal de la empresa. \\
Tecnológico &  Son aquellos provocados por fallos en la tecnología.\\
Natural &  Son aquellos provocados por desastres naturales (incendios, inundaciones, etc).\\
Externo &  Son aquellos provocados por factores externos a la empresa.\\
Interno & Son aquellos provocados por factores internos a la empresa. \\ \hline
\end{tabular}
\caption{Descripción de los riesgos según su origen}
\label{tab:origenRiesgos}
\end{center}
\end{table}




\par En las tablas que se muestran a continuación, hemos realizado una clasificación de los tipos de los riesgos que pueden afectar al proyecto según su tipo y teniendo en cuenta el nivel de pérdidas que tienen asociados.

\begin{table}[H]
\begin{center}
\begin{tabular}{p{3,5cm} p{11cm}}
\textbf{Nombre} & \textbf{Descripción}  \\ \hline
Riesgos del proyecto & Identifican problemas potenciales del proyecto (presupuesto, plazos de entrega, personal, recursos, etc.).\\
Riesgos técnicos &  Identificación de posibles problemas tales como ambigüedad en la especificación, diseño, implementación , etc.\\
Riesgos del negocio &  Identificación de riesgos del mercado. \\ \hline
\end{tabular}
\caption{Clasificación de los riesgos}
\label{clasificacionRiesgos}
\end{center}
\end{table}

\par Estos tipos de riesgos podemos clasificarlos en dos grandes grupos con la finalidad de entenderlos mejor:

\begin{itemize}[-]
  \item \textbf{Primer Grupo:} este grupo lo compondrían los riesgos impredecibles, que son aquellos que pueden ocurrir pero es complicado identificarlos por adelantado.
  \item \textbf{Segundo Grupo:} este grupo lo compondrían los riesgos predecibles, aquellos que se pueden predecir después de una evaluación detallada del plan de proyecto o que se obtienen de la experiencia del equipo en proyectos anteriores.
\end{itemize}

\subsubsection{Definición de Parámetros de los Riesgos}
\par A continuación se enumeran los parámetros utilizados para representar los posibles riesgos:

\begin{itemize}[-]
  \item \textbf{Identificador del riesgo:} formado por “Riesgo” seguido de un guión y el número identificativo del riesgo. Por ejemplo, “Riesgo-XX”.
  \item \textbf{Nombre del riesgo:} campo identificativo que da una breve idea del tipo de riesgo que se está explicando.
  \item \textbf{Origen del riesgo:} procedencia del riesgo. Se utilizará la tabla \ref{tab:origenRiesgos}.
  \item \textbf{Probabilidad de ocurrencia:} porcentaje de aparición de dichos riesgos.
  \item \textbf{Impacto del riesgo:} grado de repercusión del riesgo (muy bajo, bajo, moderado, alto, muy alto).
  \item \textbf{Descripción:} explicación detallada del riesgo a tratar.
  \item \textbf{Consecuencias:} explicación de los efectos que produciría dicho riesgo.
\end{itemize}

\subsubsection{Identificación de los Riesgos}
\par Los riesgos que pueden identificar al proyecto son los siguientes:

\begin{table}[H]
\begin{center}
\begin{tabular}{p{5,10cm} p{7cm}}
\multicolumn{2}{c}{\textbf{Riesgo-01} } \\
\hline \hline
\textbf{Nombre del riesgo} & Inundación \\
\hline
\textbf{Origen del riesgo} & Natural \\
\hline
\textbf{Probabilidad de ocurrencia} & 0.1\%  \\
\hline
\textbf{Impacto del riesgo} &  Bajo \\
\hline
\textbf{Descripción} &  Una inundación puede ser producida por unas lluvias torrenciales, la rotura de una cañería… \\
\hline
\textbf{Consecuencias} &  Daños  graves  en los equipos informáticos. En ocasiones, daños humanos. \\
\hline
\end{tabular}
\caption{Riesgo-01}
\label{tab:Riesgo-01}
\end{center}
\end{table}

\begin{table}[H]
\begin{center}
\begin{tabular}{p{5,10cm} p{7cm}}
\multicolumn{2}{c}{\textbf{Riesgo-02} } \\
\hline \hline
\textbf{Nombre del riesgo} &  Incendio\\
\hline
\textbf{Origen del riesgo} & Natural\\
\hline
\textbf{Probabilidad de ocurrencia} &  2\% \\
\hline
\textbf{Impacto del riesgo} &  Moderado \\
\hline
\textbf{Descripción} & Un incendio puede ser provocado como consecuencia de un cortocircuito, un descuido humano o de forma intencionada.  \\
\hline
\textbf{Consecuencias} & Destrucción de la zona de trabajo. En ocasiones daños humanos.  \\
\hline
\end{tabular}
\caption{Riesgo-02}
\label{tab:Riesgo-02}
\end{center}
\end{table}

\begin{table}[H]
\begin{center}
\begin{tabular}{p{5,10cm} p{7cm}}
\multicolumn{2}{c}{\textbf{Riesgo-03} } \\
\hline \hline
\textbf{Nombre del riesgo} & Terremoto \\
\hline
\textbf{Origen del riesgo} & Natural\\
\hline
\textbf{Probabilidad de ocurrencia} &1\%\\
\hline
\textbf{Impacto del riesgo} & Alto  \\
\hline
\textbf{Descripción} & Movimiento brusco de la tierra causado por el rozamiento de placas tectónicas.\\
\hline
\textbf{Consecuencias} & Daños graves de la zona de trabajo. En ocasiones daños humanos.  \\
\hline
\end{tabular}
\caption{Riesgo-03}
\label{tab:Riesgo-03}
\end{center}
\end{table}

\begin{table}[H]
\begin{center}
\begin{tabular}{p{5,10cm} p{7cm}}
\multicolumn{2}{c}{\textbf{Riesgo-04} } \\
\hline \hline
\textbf{Nombre del riesgo} &  Problema eléctrico\\
\hline
\textbf{Origen del riesgo} & Natural\\
\hline
\textbf{Probabilidad de ocurrencia} & 6\%\\
\hline
\textbf{Impacto del riesgo} &  Alto \\
\hline
\textbf{Descripción} & Problema que puede surgir de la instalación eléctrica, como por ejemplo una sobrecarga o un cortocircuito.\\
\hline
\textbf{Consecuencias} &  Daños graves en los equipos informáticos. Puede dejar a la zona de trabajo sin conexión a la red. \\
\hline
\end{tabular}
\caption{Riesgo-04}
\label{tab:Riesgo-04}
\end{center}
\end{table}

\begin{table}[H]
\begin{center}
\begin{tabular}{p{5,10cm} p{7cm}}
\multicolumn{2}{c}{\textbf{Riesgo-05} } \\
\hline \hline
\textbf{Nombre del riesgo} & Fallo en la comunicación con el cliente \\
\hline
\textbf{Origen del riesgo} & Personal\\
\hline
\textbf{Probabilidad de ocurrencia} & 65\%  \\
\hline
\textbf{Impacto del riesgo} & Moderado  \\
\hline
\textbf{Descripción} & Falta de entendimiento con el cliente  \\
\hline
\textbf{Consecuencias} & Puede provocar requisitos erróneos y retrasos en las entregas.  \\
\hline
\end{tabular}
\caption{Riesgo-05}
\label{tab:Riesgo-05}
\end{center}
\end{table}

\begin{table}[H]
\begin{center}
\begin{tabular}{p{5,10cm} p{7cm}}
\multicolumn{2}{c}{\textbf{Riesgo-06} } \\
\hline \hline
\textbf{Nombre del riesgo} & Baja definitiva. \\
\hline
\textbf{Origen del riesgo} & Personal\\
\hline
\textbf{Probabilidad de ocurrencia} &10\%\\
\hline
\textbf{Impacto del riesgo} &  Alto \\
\hline
\textbf{Descripción} &  Ocurre cuando un trabajador deja de formar parte del equipo, ya sea por finalización del contrato,  despido del empleado o fallecimiento. \\
\hline
\textbf{Consecuencias} &  Provoca que el resto del equipo tenga que hacer las tareas que el empleado tenga asignadas. También puede provocar retrasos en las entregas. \\
\hline
\end{tabular}
\caption{Riesgo-06}
\label{tab:Riesgo-06}
\end{center}
\end{table}

\begin{table}[H]
\begin{center}
\begin{tabular}{p{5,10cm} p{7cm}}
\multicolumn{2}{c}{\textbf{Riesgo-07} } \\
\hline \hline
\textbf{Nombre del riesgo} & Baja temporal \\
\hline
\textbf{Origen del riesgo} & Personal\\
\hline
\textbf{Probabilidad de ocurrencia} &  35\% \\
\hline
\textbf{Impacto del riesgo} &  Moderado \\
\hline
\textbf{Descripción} &  Ocurre cuando un trabajador se ausenta en el trabajo de forma limitada. \\
\hline
\textbf{Consecuencias} &  Provoca que el resto del equipo tenga que hacer las tareas que el empleado tenga asignadas. También puede provocar retrasos en las entregas. \\
\hline
\end{tabular}
\caption{Riesgo-07}
\label{tab:Riesgo-07}
\end{center}
\end{table}

\begin{table}[H]
\begin{center}
\begin{tabular}{p{5,10cm} p{7cm}}
\multicolumn{2}{c}{\textbf{Riesgo-08} } \\
\hline \hline
\textbf{Nombre del riesgo} & Fallo en la comunicación interna. \\
\hline
\textbf{Origen del riesgo} & Personal\\
\hline
\textbf{Probabilidad de ocurrencia} &  25\% \\
\hline
\textbf{Impacto del riesgo} &  Moderado \\
\hline
\textbf{Descripción} &  Falta de entendimiento entre los integrantes del equipo de trabajo. \\
\hline
\textbf{Consecuencias} &  Puede provocar requisitos erróneos y retrasos en las entregas. \\
\hline
\end{tabular}
\caption{Riesgo-08}
\label{tab:Riesgo-08}
\end{center}
\end{table}

\begin{table}[H]
\begin{center}
\begin{tabular}{p{5,10cm} p{7cm}}
\multicolumn{2}{c}{\textbf{Riesgo-09} } \\
\hline \hline
\textbf{Nombre del riesgo} & Mala planificación. \\
\hline
\textbf{Origen del riesgo} & Interno\\
\hline
\textbf{Probabilidad de ocurrencia} &  10\% \\
\hline
\textbf{Impacto del riesgo} &  Moderado \\
\hline
\textbf{Descripción} &  Tiene lugar cuando se hace una planificación errónea. \\
\hline
\textbf{Consecuencias} & Puede provocar empleados sobrecargados de trabajo y retrasos en las entregas.  \\
\hline
\end{tabular}
\caption{Riesgo-09}
\label{tab:Riesgo-09}
\end{center}
\end{table}

\begin{table}[H]
\begin{center}
\begin{tabular}{p{5,10cm} p{7cm}}
\multicolumn{2}{c}{\textbf{Riesgo-10} } \\
\hline \hline
\textbf{Nombre del riesgo} & Definición de requisitos errónea. \\
\hline
\textbf{Origen del riesgo} & Interno\\
\hline
\textbf{Probabilidad de ocurrencia} &   35\% \\
\hline
\textbf{Impacto del riesgo} &  Alto \\
\hline
\textbf{Descripción} &  Error en la definición de requisitos, ya sea por una mala comunicación con el cliente o interna. \\
\hline
\textbf{Consecuencias} &  Provoca requisitos erróneos y retrasos en las entregas. \\
\hline
\end{tabular}
\caption{Riesgo-10}
\label{tab:Riesgo-10}
\end{center}
\end{table}

\begin{table}[H]
\begin{center}
\begin{tabular}{p{5,10cm} p{7cm}}
\multicolumn{2}{c}{\textbf{Riesgo-11} } \\
\hline \hline
\textbf{Nombre del riesgo} & Ataque a la empresa. \\
\hline
\textbf{Origen del riesgo} & Externo\\
\hline
\textbf{Probabilidad de ocurrencia} & 5\%  \\
\hline
\textbf{Impacto del riesgo} & Moderado\\
\hline
\textbf{Descripción} &  Se produce cuando un tercero provoca daños en el inmovilizado de la empresa, de forma intencionada o voluntaria. \\
\hline
\textbf{Consecuencias} &  Puede provocar numerosos gastos en reparar o reemplazar el inmovilizado afectado. Además, puede producir retrasos en las entregas e incluso daños humanos. \\
\hline
\end{tabular}
\caption{Riesgo-11}
\label{tab:Riesgo-11}
\end{center}
\end{table}

\begin{table}[H]
\begin{center}
\begin{tabular}{p{5,10cm} p{7cm}}
\multicolumn{2}{c}{\textbf{Riesgo-12} } \\
\hline \hline
\textbf{Nombre del riesgo} &  Caída de los servidores.\\
\hline
\textbf{Origen del riesgo} & Tecnológico\\
\hline
\textbf{Probabilidad de ocurrencia} &  30\% \\
\hline
\textbf{Impacto del riesgo} &  Bajo \\
\hline
\textbf{Descripción} & Se produce cuando se pierde la conexión con los servidores con los que se trabaja.  \\
\hline
\textbf{Consecuencias} &  En función de la duración de este suceso, se pueden dar retrasos en las entregas y alteración en la planificación del proyecto. \\
\hline
\end{tabular}
\caption{Riesgo-12}
\label{tab:Riesgo-12}
\end{center}
\end{table}

\begin{table}[H]
\begin{center}
\begin{tabular}{p{5,10cm} p{7cm}}
\multicolumn{2}{c}{\textbf{Riesgo-13} } \\
\hline \hline
\textbf{Nombre del riesgo} &  Caída de la conexión a internet.\\
\hline
\textbf{Origen del riesgo} & Tecnológico\\
\hline
\textbf{Probabilidad de ocurrencia} &  5\% \\
\hline
\textbf{Impacto del riesgo} &  Bajo \\
\hline
\textbf{Descripción} &  Ocurre cuando se pierde el acceso a internet desde la zona de trabajo. \\
\hline
\textbf{Consecuencias} &  En función de la duración de este suceso, se pueden dar retrasos en las entregas y alteración en la planificación del proyecto. \\
\hline
\end{tabular}
\caption{Riesgo-13}
\label{tab:Riesgo-13}
\end{center}
\end{table}

\begin{table}[H]
\begin{center}
\begin{tabular}{p{5,10cm} p{7cm}}
\multicolumn{2}{c}{\textbf{Riesgo-14} } \\
\hline \hline
\textbf{Nombre del riesgo} & Mala estimación. \\
\hline
\textbf{Origen del riesgo} & Interno\\
\hline
\textbf{Probabilidad de ocurrencia} &60\%\\
\hline
\textbf{Impacto del riesgo} &  Moderado \\
\hline
\textbf{Descripción} & Este riesgo se debe a una estimación que no se corresponde perfectamente con el proyecto.  \\
\hline
\textbf{Consecuencias} & Puede provocar empleados sobrecargados de trabajo y retrasos en las entregas.  \\
\hline
\end{tabular}
\caption{Riesgo-14}
\label{tab:Riesgo-14}
\end{center}
\end{table}

\begin{table}[H]
\begin{center}
\begin{tabular}{p{5,10cm} p{7cm}}
\multicolumn{2}{c}{\textbf{Riesgo-15} } \\
\hline \hline
\textbf{Nombre del riesgo} & Mala selección de personal. \\
\hline
\textbf{Origen del riesgo} & Interno\\
\hline
\textbf{Probabilidad de ocurrencia} & 15\%  \\
\hline
\textbf{Impacto del riesgo} &  Moderado \\
\hline
\textbf{Descripción} &  Se produce cuando el personal contratado para el proyecto no tenga las cualidades esperadas. \\
\hline
\textbf{Consecuencias} & Puede provocar retrasos en las entregas.  \\
\hline
\end{tabular}
\caption{Riesgo-15}
\label{tab:Riesgo-15}
\end{center}
\end{table}

\begin{table}[H]
\begin{center}
\begin{tabular}{p{5,10cm} p{7cm}}
\multicolumn{2}{c}{\textbf{Riesgo-16} } \\
\hline \hline
\textbf{Nombre del riesgo} & Presupuesto erróneo \\
\hline
\textbf{Origen del riesgo} & Interno\\
\hline
\textbf{Probabilidad de ocurrencia} &  13\% \\
\hline
\textbf{Impacto del riesgo} & Moderado  \\
\hline
\textbf{Descripción} &   El presupuesto está mal calculado\\
\hline
\textbf{Consecuencias} &  Puede provocar una disminución del beneficio, e incluso pérdidas. En cualquier caso habría que volver a recalcular, y produciría un retraso en el resto del proyecto.  \\
\hline
\end{tabular}
\caption{Riesgo-16}
\label{tab:Riesgo-16}
\end{center}
\end{table}

\begin{table}[H]
\begin{center}
\begin{tabular}{p{5,10cm} p{7cm}}
\multicolumn{2}{c}{\textbf{Riesgo-17} } \\
\hline \hline
\textbf{Nombre del riesgo} & Cambios en la tecnología \\
\hline
\textbf{Origen del riesgo} & Externo\\
\hline
\textbf{Probabilidad de ocurrencia} &  15\% \\
\hline
\textbf{Impacto del riesgo} &  Bajo \\
\hline
\textbf{Descripción} &   Durante el proceso de desarrollo pueden aparecen nuevas tecnologías que sean necesarias y no se hayan tenido en cuenta. \\
\hline
\textbf{Consecuencias} &  Puede provocar un retraso en la fecha de entrega del proyecto, además de aumentar los costes para adquirir dicha tecnología. Por otra parte, introducir tecnología nueva implica aprender a usarla, o a contratar a personal que sea experto en dicha tecnología. \\
\hline
\end{tabular}
\caption{Riesgo-17}
\label{tab:Riesgo-17}
\end{center}
\end{table}

\begin{table}[H]
\begin{center}
\begin{tabular}{p{5,10cm} p{7cm}}
\multicolumn{2}{c}{\textbf{Riesgo-18} } \\
\hline \hline
\textbf{Nombre del riesgo} & Diseño no satisfactorio \\
\hline
\textbf{Origen del riesgo} & Interno\\
\hline
\textbf{Probabilidad de ocurrencia} &  12\% \\
\hline
\textbf{Impacto del riesgo} & Muy Alto  \\
\hline
\textbf{Descripción} &  El diseño podría contener errores y por lo tanto afectar al correcto funcionamiento del sistema.  \\
\hline
\textbf{Consecuencias} &  Volver a diseñar las partes que afecten al sistema.  \\
\hline
\end{tabular}
\caption{Riesgo-18}
\label{tab:Riesgo-18}
\end{center}
\end{table}

\begin{table}[H]
\begin{center}
\begin{tabular}{p{5,10cm} p{7cm}}
\multicolumn{2}{c}{\textbf{Riesgo-19} } \\
\hline \hline
\textbf{Nombre del riesgo} &  Retrasos por parte del cliente \\
\hline
\textbf{Origen del riesgo} & Externo\\
\hline
\textbf{Probabilidad de ocurrencia} &  22\% \\
\hline
\textbf{Impacto del riesgo} &  Moderado \\
\hline
\textbf{Descripción} & El cliente puede retrasarse en la aprobación de las fases del proyecto.   \\
\hline
\textbf{Consecuencias} & Retrasos en el desarrollo del proyecto en todas aquellas partes que requieran la aprobación del cliente que falta.   \\
\hline
\end{tabular}
\caption{Riesgo-19}
\label{tab:Riesgo-19}
\end{center}
\end{table}

\begin{table}[H]
\begin{center}
\begin{tabular}{p{5,10cm} p{7cm}}
\multicolumn{2}{c}{\textbf{Riesgo-20} } \\
\hline \hline
\textbf{Nombre del riesgo} &  Quiebra Smart Software Solutions\\
\hline
\textbf{Origen del riesgo} & Interno\\
\hline
\textbf{Probabilidad de ocurrencia} &  10\% \\
\hline
\textbf{Impacto del riesgo} &  Muy Alto \\
\hline
\textbf{Descripción} &  La propia empresa desarrolladora del proyecto puede entrar en quiebra.  \\
\hline
\textbf{Consecuencias} & El trabajo realizado no serviría de nada, el proyecto finalizará.   \\
\hline
\end{tabular}
\caption{Riesgo-20}
\label{tab:Riesgo-20}
\end{center}
\end{table}

\begin{table}[H]
\begin{center}
\begin{tabular}{p{5,10cm} p{7cm}}
\multicolumn{2}{c}{\textbf{Riesgo-21} } \\
\hline \hline
\textbf{Nombre del riesgo} & Quiebra en el cliente \\
\hline
\textbf{Origen del riesgo} & Externo\\
\hline
\textbf{Probabilidad de ocurrencia} &  15\% \\
\hline
\textbf{Impacto del riesgo} &  Muy Alto \\
\hline
\textbf{Descripción} & Durante el desarrollo del proyecto, el cliente puede entrar en quiebra, y cancelar el proyecto.   \\
\hline
\textbf{Consecuencias} &  El trabajo realizado no serviría de nada, el proyecto finalizará, y se intentará recuperar parte del dinero.  \\
\hline
\end{tabular}
\caption{Riesgo-21}
\label{tab:Riesgo-21}
\end{center}
\end{table}

\begin{table}[H]
\begin{center}
\begin{tabular}{p{5,10cm} p{7cm}}
\multicolumn{2}{c}{\textbf{Riesgo-22} } \\
\hline \hline
\textbf{Nombre del riesgo} & Campos electromagnéticos  \\
\hline
\textbf{Origen del riesgo} & Interno/Externo\\
\hline
\textbf{Probabilidad de ocurrencia} &  25\% \\
\hline
\textbf{Impacto del riesgo} &  Moderado \\
\hline
\textbf{Descripción} & Es posible que los equipos informáticos dejen de funcionar debido a la formación de campos electromagnéticos provocados por entrar en contacto con radares u otros equipos.   \\
\hline
\textbf{Consecuencias} &  Puede producir un incremento en els coste al solucionar la avería, así como retrasar al equipo de trabajo. Además, podemos perder información importante.  \\
\hline
\end{tabular}
\caption{Riesgo-22}
\label{tab:Riesgo-22}
\end{center}
\end{table}

\begin{table}[H]
\begin{center}
\begin{tabular}{p{5,10cm} p{7cm}}
\multicolumn{2}{c}{\textbf{Riesgo-23} } \\
\hline \hline
\textbf{Nombre del riesgo} & Avería en las cámaras \\
\hline
\textbf{Origen del riesgo} & Interno\\
\hline
\textbf{Probabilidad de ocurrencia} & 8\%  \\
\hline
\textbf{Impacto del riesgo} &  Alto \\
\hline
\textbf{Descripción} &  Durante el acoplamiento de las cámaras  en el vehículo, puede suceder que tengan  algún tipo de avería que afecte al correcto funcionamiento del software.  \\
\hline
\textbf{Consecuencias} & Puede producir un incremento en els coste al solucionar la avería, así como retrasar al equipo de trabajo.   \\
\hline
\end{tabular}
\caption{Riesgo-23}
\label{tab:Riesgo-23}
\end{center}
\end{table}

\begin{table}[H]
\begin{center}
\begin{tabular}{p{5,10cm} p{7cm}}
\multicolumn{2}{c}{\textbf{Riesgo-24} } \\
\hline \hline
\textbf{Nombre del riesgo} & Avería en la antena GPS \\
\hline
\textbf{Origen del riesgo} & Interno\\
\hline
\textbf{Probabilidad de ocurrencia} &  8\% \\
\hline
\textbf{Impacto del riesgo} &  Alto \\
\hline
\textbf{Descripción} & Durante el acoplamiento de la antena GPS en el vehículo, puede suceder que tengan  algún tipo de avería que afecte al correcto funcionamiento del software.   \\
\hline
\textbf{Consecuencias} &  Puede producir un incremento en els coste al solucionar la avería, así como retrasar al equipo de trabajo.  \\
\hline
\end{tabular}
\caption{Riesgo-24}
\label{tab:Riesgo-24}
\end{center}
\end{table}

\begin{table}[H]
\begin{center}
\begin{tabular}{p{5,10cm} p{7cm}}
\multicolumn{2}{c}{\textbf{Riesgo-25} } \\
\hline \hline
\textbf{Nombre del riesgo} & Avería en el sensor de distancia  \\
\hline
\textbf{Origen del riesgo} & Interno\\
\hline
\textbf{Probabilidad de ocurrencia} &  8\% \\
\hline
\textbf{Impacto del riesgo} &  Alto \\
\hline
\textbf{Descripción} &  Durante el acoplamiento del sensor de distancia en el vehículo, puede suceder que tengan  algún tipo de avería que afecte al correcto funcionamiento del software. \\
\hline
\textbf{Consecuencias} & Puede producir un incremento en els coste al solucionar la avería, así como retrasar al equipo de trabajo.   \\
\hline
\end{tabular}
\caption{Riesgo-25}
\label{tab:Riesgo-25}
\end{center}
\end{table}

\begin{table}[H]
\begin{center}
\begin{tabular}{p{5,10cm} p{7cm}}
\multicolumn{2}{c}{\textbf{Riesgo-26} } \\
\hline \hline
\textbf{Nombre del riesgo} & Avería en el hardware de control \\
\hline
\textbf{Origen del riesgo} & Interno \\
\hline
\textbf{Probabilidad de ocurrencia} & 8\%  \\
\hline
\textbf{Impacto del riesgo} &  Alto \\
\hline
\textbf{Descripción} &  Durante el acoplamiento de la Raspberry y el Arduino en el vehículo, puede suceder que tengan  algún tipo de avería que afecte al correcto funcionamiento del software. \\
\hline
\textbf{Consecuencias} &  Puede producir un incremento en els coste al solucionar la avería, así como retrasar al equipo de trabajo.  \\
\hline
\end{tabular}
\caption{Riesgo-26}
\label{tab:Riesgo-26}
\end{center}
\end{table}

\subsection{Análisis de Riesgos}
\subsubsection{Análisis Cualitativo}
\par El análisis cualitativo estima, para cada uno de los riesgos anteriores, el impacto y la probabilidad de aparecer. Con esta clasificación conseguimos saber si un riesgo requiere una respuesta inmediata o, por el contrario, se le puede conceder menos prioridad a arreglar otros riesgos. En resumen, la siguiente tabla muestra qué impactos afectan a qué partes del proyecto y su gravedad:

\begin{table}[H]
\begin{center}
\begin{tabular}{p{2cm}|p{2.5cm} p{2.5cm} p{2.5cm} p{2.5cm} p{2.5cm}}
   & \textbf{Muy Bajo} & \textbf{Bajo} & \textbf{Moderado} & \textbf{Alto} & \textbf{Muy Alto}\\
\hline \hline
\textbf{Coste} & Cambio en el coste inapreciable (<5\%) & Aumento del coste entre el 5\% y el 35\% del margen de riesgos.& Incremento de costes entre el 35\% y el 65\%.& Incremento de costes igual al margen de riesgos.&El coste supera el margen del presupuesto destinado a los riesgos. \\
\hline
\textbf{Calendario} & La planificación no cambiará.& Pequeño retraso en la entrega (no mayor a tres días).&El retraso será moderado (de al menos una semana completa). & El proyecto se verá retrasado 2 semanas al menos.& El proyecto debe volver a planificarse (un mes o más).\\
\hline
\textbf{Alcance} & No afecta al alcance del proyecto.& Las partes del proyecto que se ven afectadas son secundarias.& Las partes del proyecto que se ven afectadas son secundarias.&La mayoría de las partes principales del proyecto son afectadas. & Proyecto descartado por el cliente.\\
\hline
\textbf{Calidad} & La calidad del producto no se verá afectada.& Algunas partes del producto verán alterada su calidad.& Las partes del proyecto que se ven afectadas son secundarias.& El cliente no acepta la reducción de calidad.& La calidad no es aceptada por el cliente ni por los responsables de calidad de  la empresa. \\
\hline
\end{tabular}
\caption{Análisis Cualitativo: Impacto de los objetivos}
\label{tab:analisis-cualitativo}
\end{center}
\end{table}


\subsubsection{Análisis Cuantitativo}
\par En este punto se analizarán los riesgos descritos asociando un valor numérico a cada uno de los grados de impacto, con el fin de facilitar el entendimiento y el impacto frente a la probabilidad de cada riesgo.
\par Asociación de valores a los grados de impacto:
\begin{itemize}[-]
  \item \textbf{Muy Bajo:} 2
  \item \textbf{Bajo:} 4
  \item \textbf{Moderado:} 6
  \item \textbf{Alto:} 8
  \item \textbf{Muy Alto:} 10
\end{itemize}

\par A continuación realizaremos un cálculo para conseguir un factor de riesgo. Este factor de riesgo se calcula multiplicando el valor de los grados de impacto por el porcentaje de ocurrencia de los mismos. Esto se realizará para cada uno de los riesgos descritos anteriormente. Este factor será el dato que usaremos para cuantificar los riesgos.


\begin{table}[H]
\begin{center}
\begin{tabular}{c|ccc}
\textbf{Riesgo} & \textbf{Probabilidad} & \textbf{Impacto} & \textbf{Factor de riesgo} \\
\hline \hline
Riesgo-01	&	0,001	&	4	&	0,004	\\
Riesgo-02	&	0,020	&	6	&	0,120	\\
Riesgo-03	&	0,010	&	8	&	0,080	\\
Riesgo-04	&	0,060	&	8	&	0,480	\\
Riesgo-05	&	0,650	&	6	&	3,900	\\
Riesgo-06	&	0,350	&	6	&	2,100	\\
Riesgo-07	&	0,100	&	8	&	0,800	\\
Riesgo-08	&	0,250	&	6	&	1,500	\\
Riesgo-09	&	0,100	&	6	&	0,600	\\
Riesgo-10	&	0,350	&	8	&	2,800	\\
Riesgo-11	&	0,050	&	6	&	0,300	\\
Riesgo-12	&	0,300	&	4	&	1,200	\\
Riesgo-13	&	0,050	&	4	&	0,200	\\
Riesgo-14	&	0,650	&	6	&	3,900	\\
Riesgo-15	&	0,150	&	6	&	0,900	\\
Riesgo-16	&	0,130	&	6	&	0,780	\\
Riesgo-17	&	0,150	&	4	&	0,600	\\
Riesgo-18	&	0,120	&	10	&	1,200	\\
Riesgo-19	&	0,220	&	6	&	1,320	\\
Riesgo-20	&	0,100	&	10	&	1,000	\\
Riesgo-21	&	0,150	&	10	&	1,500	\\
Riesgo-22	&	0,250	&	6	&	1,500	\\
Riesgo-23	&	0,080	&	8	&	0,640	\\
Riesgo-24	&	0,080	&	8	&	0,640	\\
Riesgo-25	&	0,080	&	8	&	0,640	\\
Riesgo-26	&	0,080	&	8	&	0,640	\\ \hline
\end{tabular}
\caption{Factor de riesgo}
\label{tab:analisis-cualitativo}
\end{center}
\end{table}


\subsection{Prevención de Riesgos y Elaboración del Plan de Contingencia}

\par A continuación se detallan para cada uno de los riesgos, el plan diseñado para prevenirlos y el plan de recuperación por si finalmente ocurren.

\begin{table}[H]
\begin{center}
\begin{tabular}{p{5,3cm} p{10cm}}
\multicolumn{2}{c}{\textbf{Riesgo-01} } \\
\hline \hline
\textbf{Nombre del riesgo} & Inundación \\
\hline
\textbf{Prevención de riesgos} & \begin{itemize}[-]
  \item Realizar un cableado en el lugar de trabajo en sitios altos.
  \item No tener deteriorado el sistema de desagües.
  \item No dejar material importante en el suelo o cerca de lavabos.
  \item Tener contratado un seguro contra inundaciones.
  \end{itemize} \\
\hline
\textbf{Plan de recuperación} &   \begin{itemize}[-]
  \item Realizar un inventario de los daños materiales.
  \item Recuperar las últimas versiones de los documentos.
  \item Reponer los equipos rotos.
  \end{itemize}\\
\hline
\end{tabular}
\caption{Plan de prevención y recuperación Riesgo-01}
\label{tab:Riesgo-01-Prev_Recup}
\end{center}
\end{table}

\begin{table}[H]
\begin{center}
\begin{tabular}{p{5,3cm} p{10cm}}
\multicolumn{2}{c}{\textbf{Riesgo-02} } \\
\hline \hline
\textbf{Nombre del riesgo} & Incendio \\
\hline
\textbf{Prevención de riesgos} & \begin{itemize}[-]
  \item Dotar al lugar de trabajo con extintores, alarma de incendios y salidas de emergencia.
  \item Revisiones periódicas del cableado del edificio.
  \item Tener contratado un seguro contra incendios.
  \item Almacenar en lugar seguro el material importante.
  \end{itemize} \\
\hline
\textbf{Plan de recuperación} &   \begin{itemize}[-]
  \item Realizar un inventario de los daños materiales.
  \item Recuperar las últimas versiones de los documentos.
  \item Reponer los equipos rotos.
  \end{itemize}\\
\hline
\end{tabular}
\caption{Plan de prevención y recuperación Riesgo-02}
\label{tab:Riesgo-02-Prev_Recup}
\end{center}
\end{table}

\begin{table}[H]
\begin{center}
\begin{tabular}{p{5,3cm} p{10cm}}
\multicolumn{2}{c}{\textbf{Riesgo-03} } \\
\hline \hline
\textbf{Nombre del riesgo} & Terremoto \\
\hline
\textbf{Prevención de riesgos} & \begin{itemize}[-]
  \item Asegurar los equipos informáticos para que no queden sueltos y puedan caer.
  \end{itemize} \\
\hline
\textbf{Plan de recuperación} &   \begin{itemize}[-]
  \item Realizar un inventario de los daños materiales.
  \item Recuperar las últimas versiones de los documentos.
  \item Reponer los equipos rotos.
  \end{itemize}\\
\hline
\end{tabular}
\caption{Plan de prevención y recuperación Riesgo-03}
\label{tab:Riesgo-03-Prev_Recup}
\end{center}
\end{table}

\begin{table}[H]
\begin{center}
\begin{tabular}{p{5,3cm} p{10cm}}
\multicolumn{2}{c}{\textbf{Riesgo-04} } \\
\hline \hline
\textbf{Nombre del riesgo} & Problema eléctrico \\
\hline
\textbf{Prevención de riesgos} & \begin{itemize}[-]
  \item Tener un sistema eléctrico separado en circuitos independientes.
  \item Hacer revisiones periódicas del sistema eléctrico
  \end{itemize} \\
\hline
\textbf{Plan de recuperación} &   \begin{itemize}[-]
  \item Realizar una evaluación de los daños.
  \item Reponer los equipos estropeados.
  \end{itemize}\\
\hline
\end{tabular}
\caption{Plan de prevención y recuperación Riesgo-04}
\label{tab:Riesgo-04-Prev_Recup}
\end{center}
\end{table}

\begin{table}[H]
\begin{center}
\begin{tabular}{p{5,3cm} p{10cm}}
\multicolumn{2}{c}{\textbf{Riesgo-05} } \\
\hline \hline
\textbf{Nombre del riesgo} & Fallo en la comunicación con el cliente \\
\hline
\textbf{Prevención de riesgos} & \begin{itemize}[-]
  \item Mantener reuniones periódicas con el cliente para informarle del estado del proyecto.
  \end{itemize} \\
\hline
\textbf{Plan de recuperación} &   \begin{itemize}[-]
  \item Reunirse con el cliente para evaluar la disonancia.
  \end{itemize}\\
\hline
\end{tabular}
\caption{Plan de prevención y recuperación Riesgo-05}
\label{tab:Riesgo-05-Prev_Recup}
\end{center}
\end{table}

\begin{table}[H]
\begin{center}
\begin{tabular}{p{5,3cm} p{10cm}}
\multicolumn{2}{c}{\textbf{Riesgo-06} } \\
\hline \hline
\textbf{Nombre del riesgo} & Baja definitiva \\
\hline
\textbf{Prevención de riesgos} & \begin{itemize}[-]
  \item Cuidar la salud de los empleados.
  \item Revisar la duración de los contratos del equipo de trabajo.
  \end{itemize} \\
\hline
\textbf{Plan de recuperación} &   \begin{itemize}[-]
  \item Reasignar las tareas a otros empleados.
  \item Contratar a otro empleado.
  \end{itemize}\\
\hline
\end{tabular}
\caption{Plan de prevención y recuperación Riesgo-06}
\label{tab:Riesgo-06-Prev_Recup}
\end{center}
\end{table}


\begin{table}[H]
\begin{center}
\begin{tabular}{p{5,3cm} p{10cm}}
\multicolumn{2}{c}{\textbf{Riesgo-07} } \\
\hline \hline
\textbf{Nombre del riesgo} & Baja temporal \\
\hline
\textbf{Prevención de riesgos} & \begin{itemize}[-]
  \item Cuidar la salud de los empleados.
  \end{itemize} \\
\hline
\textbf{Plan de recuperación} &   \begin{itemize}[-]
  \item Reasignar las tareas a otros empleados.
  \end{itemize}\\
\hline
\end{tabular}
\caption{Plan de prevención y recuperación Riesgo-07}
\label{tab:Riesgo-07-Prev_Recup}
\end{center}
\end{table}



\begin{table}[H]
\begin{center}
\begin{tabular}{p{5,3cm} p{10cm}}
\multicolumn{2}{c}{\textbf{Riesgo-08} } \\
\hline \hline
\textbf{Nombre del riesgo} & Fallo en la comunicación interna \\
\hline
\textbf{Prevención de riesgos} & \begin{itemize}[-]
  \item Realizar reuniones de equipo regularmente.
  \end{itemize} \\
\hline
\textbf{Plan de recuperación} &   \begin{itemize}[-]
  \item Realizar una reunión para evaluar los fallos y solucionarlos.
  \end{itemize}\\
\hline
\end{tabular}
\caption{Plan de prevención y recuperación Riesgo-08}
\label{tab:Riesgo-08-Prev_Recup}
\end{center}
\end{table}

\begin{table}[H]
\begin{center}
\begin{tabular}{p{5,3cm} p{10cm}}
\multicolumn{2}{c}{\textbf{Riesgo-09} } \\
\hline \hline
\textbf{Nombre del riesgo} & Mala planificación \\
\hline
\textbf{Prevención de riesgos} & \begin{itemize}[-]
  \item Los encargados de realizar la planificación deben estar bien cualificados y tener experiencia en esta tarea.
  \end{itemize} \\
\hline
\textbf{Plan de recuperación} &   \begin{itemize}[-]
  \item Realizar una nueva planificación y reunirse con el cliente para informarle de los cambios.
  \end{itemize}\\
\hline
\end{tabular}
\caption{Plan de prevención y recuperación Riesgo-09}
\label{tab:Riesgo-09-Prev_Recup}
\end{center}
\end{table}

\begin{table}[H]
\begin{center}
\begin{tabular}{p{5,3cm} p{10cm}}
\multicolumn{2}{c}{\textbf{Riesgo-10} } \\
\hline \hline
\textbf{Nombre del riesgo} & Definición de requisitos errónea \\
\hline
\textbf{Prevención de riesgos} & \begin{itemize}[-]
  \item Los encargados de realizar la planificación deben estar bien cualificados y tener experiencia en esta tarea.
  \end{itemize} \\
\hline
\textbf{Plan de recuperación} &   \begin{itemize}[-]
  \item Realizar una reunión para evaluar los fallos y solucionarlos.
  \end{itemize}\\
\hline
\end{tabular}
\caption{Plan de prevención y recuperación Riesgo-10}
\label{tab:Riesgo-10-Prev_Recup}
\end{center}
\end{table}

\begin{table}[H]
\begin{center}
\begin{tabular}{p{5,3cm} p{10cm}}
\multicolumn{2}{c}{\textbf{Riesgo-11} } \\
\hline \hline
\textbf{Nombre del riesgo} & Ataque en la empresa \\
\hline
\textbf{Prevención de riesgos} & \begin{itemize}[-]
  \item Contar con un sistema de seguridad.
  \item Realizar copias de seguridad en servidores de la nube.
  \end{itemize} \\
\hline
\textbf{Plan de recuperación} &   \begin{itemize}[-]
  \item Reemplazar el inmovilizado afectado.
  \item Restaurar la última versión guardada en el servidor.
  \end{itemize}\\
\hline
\end{tabular}
\caption{Plan de prevención y recuperación Riesgo-11}
\label{tab:Riesgo-11-Prev_Recup}
\end{center}
\end{table}

\begin{table}[H]
\begin{center}
\begin{tabular}{p{5,3cm} p{10cm}}
\multicolumn{2}{c}{\textbf{Riesgo-12} } \\
\hline \hline
\textbf{Nombre del riesgo} & Caída de los servidores \\
\hline
\textbf{Prevención de riesgos} & \begin{itemize}[-]
  \item Realizar copias de seguridad en distintos servidores.
  \item Aumentar la capacidad de los servidores si son propios o adquirir una licencia de mayor capacidad si se contratan.
  \end{itemize} \\
\hline
\textbf{Plan de recuperación} &   \begin{itemize}[-]
  \item Restaurar la última versión guardada en el servidor que no se ha caído.
  \end{itemize}\\
\hline
\end{tabular}
\caption{Plan de prevención y recuperación Riesgo-12}
\label{tab:Riesgo-12-Prev_Recup}
\end{center}
\end{table}

\begin{table}[H]
\begin{center}
\begin{tabular}{p{5,3cm} p{10cm}}
\multicolumn{2}{c}{\textbf{Riesgo-13} } \\
\hline \hline
\textbf{Nombre del riesgo} & Caída de la conexión a internet \\
\hline
\textbf{Prevención de riesgos} & \begin{itemize}[-]
  \item Cuando se hagan cambios en el proyecto guardarlos.
  \end{itemize} \\
\hline
\textbf{Plan de recuperación} &   \begin{itemize}[-]
  \item Restaurar la última versión guardada en el servidor.
  \item Contratar a otra empresa proveedora de internet.
  \end{itemize}\\
\hline
\end{tabular}
\caption{Plan de prevención y recuperación Riesgo-13}
\label{tab:Riesgo-13-Prev_Recup}
\end{center}
\end{table}

\begin{table}[H]
\begin{center}
\begin{tabular}{p{5,3cm} p{10cm}}
\multicolumn{2}{c}{\textbf{Riesgo-14} } \\
\hline \hline
\textbf{Nombre del riesgo} & Mala estimación  \\
\hline
\textbf{Prevención de riesgos} & \begin{itemize}[-]
  \item Los encargados de realizar la planificación deben estar bien cualificados y tener experiencia en esta tarea.
  \end{itemize} \\
\hline
\textbf{Plan de recuperación} &   \begin{itemize}[-]
  \item Realizar una reunión para evaluar los fallos y solucionarlos.
  \end{itemize}\\
\hline
\end{tabular}
\caption{Plan de prevención y recuperación Riesgo-14}
\label{tab:Riesgo-14-Prev_Recup}
\end{center}
\end{table}

\begin{table}[H]
\begin{center}
\begin{tabular}{p{5,3cm} p{10cm}}
\multicolumn{2}{c}{\textbf{Riesgo-15} } \\
\hline \hline
\textbf{Nombre del riesgo} & Mala selección de personal \\
\hline
\textbf{Prevención de riesgos} & \begin{itemize}[-]
  \item El equipo de recursos humanos debe estar bien cualificado.
  \item La prueba de selección debe ser precisa.
  \end{itemize} \\
\hline
\textbf{Plan de recuperación} &   \begin{itemize}[-]
  \item Formar al empleado
  \item Despedir al empleado y contratar a otro más cualificado.
  \end{itemize}\\
\hline
\end{tabular}
\caption{Plan de prevención y recuperación Riesgo-15}
\label{tab:Riesgo-15-Prev_Recup}
\end{center}
\end{table}

\begin{table}[H]
\begin{center}
\begin{tabular}{p{5,3cm} p{10cm}}
\multicolumn{2}{c}{\textbf{Riesgo-16} } \\
\hline \hline
\textbf{Nombre del riesgo} & Presupuesto erróneo  \\
\hline
\textbf{Prevención de riesgos} & \begin{itemize}[-]
  \item Los encargados de realizar la planificación deben estar bien cualificados y tener experiencia
  \end{itemize} \\
\hline
\textbf{Plan de recuperación} &   \begin{itemize}[-]
  \item Realizar una reunión para evaluar los fallos y solucionarlos.
  \item Reunirse con el cliente para informarle de los cambios realizados.
  \end{itemize}\\
\hline
\end{tabular}
\caption{Plan de prevención y recuperación Riesgo-16}
\label{tab:Riesgo-16-Prev_Recup}
\end{center}
\end{table}

\begin{table}[H]
\begin{center}
\begin{tabular}{p{5,3cm} p{10cm}}
\multicolumn{2}{c}{\textbf{Riesgo-17} } \\
\hline \hline
\textbf{Nombre del riesgo} & Cambios en la tecnología \\
\hline
\textbf{Prevención de riesgos} & \begin{itemize}[-]
  \item Estar informado de los cambios tecnológicos que pueden afectar al proyecto y evaluar si son necesarios.
  \end{itemize} \\
\hline
\textbf{Plan de recuperación} &   \begin{itemize}[-]
  \item Adquirir el software/hardware necesario
  \item Informar al cliente sobre los posibles cambios
  \end{itemize}\\
\hline
\end{tabular}
\caption{Plan de prevención y recuperación Riesgo-17}
\label{tab:Riesgo-17-Prev_Recup}
\end{center}
\end{table}

\begin{table}[H]
\begin{center}
\begin{tabular}{p{5,3cm} p{10cm}}
\multicolumn{2}{c}{\textbf{Riesgo-18} } \\
\hline \hline
\textbf{Nombre del riesgo} & Diseño no satisfactorio \\
\hline
\textbf{Prevención de riesgos} & \begin{itemize}[-]
  \item Seguir todo lo detallado en los documentos en referencia al diseño
  \item Dar apoyo al equipo de pruebas para ampliar la colección de pruebas y contemplar todos los casos posibles.
  \end{itemize} \\
\hline
\textbf{Plan de recuperación} &   \begin{itemize}[-]
  \item Rediseñar los componentes que den problemas
  \end{itemize}\\
\hline
\end{tabular}
\caption{Plan de prevención y recuperación Riesgo-18}
\label{tab:Riesgo-18-Prev_Recup}
\end{center}
\end{table}

\begin{table}[H]
\begin{center}
\begin{tabular}{p{5,3cm} p{10cm}}
\multicolumn{2}{c}{\textbf{Riesgo-19} } \\
\hline \hline
\textbf{Nombre del riesgo} & Retrasos por parte del cliente \\
\hline
\textbf{Prevención de riesgos} & \begin{itemize}[-]
  \item Mantener reuniones periódicas con el cliente
  \item Hacer descripciones más detalladas para que el cliente entienda mejor el proyecto
  \end{itemize} \\
\hline
\textbf{Plan de recuperación} &   \begin{itemize}[-]
  \item Informar al cliente sobre los retrasos que se producen.
  \end{itemize}\\
\hline
\end{tabular}
\caption{Plan de prevención y recuperación Riesgo-19}
\label{tab:Riesgo-19-Prev_Recup}
\end{center}
\end{table}

\begin{table}[H]
\begin{center}
\begin{tabular}{p{5,3cm} p{10cm}}
\multicolumn{2}{c}{\textbf{Riesgo-20} } \\
\hline \hline
\textbf{Nombre del riesgo} & Quiebra SmartSoftwareSolutions \\
\hline
\textbf{Prevención de riesgos} & \begin{itemize}[-]
  \item Antes de aceptar el proyecto, analizar la situación económica de la empresa para ver si es posible aguantar la carga económica del proyecto.
  \end{itemize} \\
\hline
\textbf{Plan de recuperación} &   \begin{itemize}[-]
  \item Buscar una empresa que pueda seguir con el proyecto propuesto por el cliente
  \end{itemize}\\
\hline
\end{tabular}
\caption{Plan de prevención y recuperación Riesgo-20}
\label{tab:Riesgo-20-Prev_Recup}
\end{center}
\end{table}

\begin{table}[H]
\begin{center}
\begin{tabular}{p{5,3cm} p{10cm}}
\multicolumn{2}{c}{\textbf{Riesgo-21} } \\
\hline \hline
\textbf{Nombre del riesgo} & Quiebra del cliente \\
\hline
\textbf{Prevención de riesgos} & \begin{itemize}[-]
  \item Antes de aceptar el proyecto, investigar la situación económica del cliente
  \end{itemize} \\
\hline
\textbf{Plan de recuperación} &   \begin{itemize}[-]
  \item BBuscar un sustituto que pueda estar interesado en el producto, y adaptarlo a sus nuevas necesidades.
  \end{itemize}\\
\hline
\end{tabular}
\caption{Plan de prevención y recuperación Riesgo-21}
\label{tab:Riesgo-21-Prev_Recup}
\end{center}
\end{table}

\begin{table}[H]
\begin{center}
\begin{tabular}{p{5,3cm} p{10cm}}
\multicolumn{2}{c}{\textbf{Riesgo-22} } \\
\hline \hline
\textbf{Nombre del riesgo} & Campos electromagnéticos \\
\hline
\textbf{Prevención de riesgos} & \begin{itemize}[-]
  \item Aislar los equipos para evitar que las señales electromagnéticas produzcan daños
  \end{itemize} \\
\hline
\textbf{Plan de recuperación} &   \begin{itemize}[-]
  \item Recuperar la información dañada de las copias de seguridad.
  \item Adquirir o arreglas los equipos afectados.
  \end{itemize}\\
\hline
\end{tabular}
\caption{Plan de prevención y recuperación Riesgo-22}
\label{tab:Riesgo-22-Prev_Recup}
\end{center}
\end{table}

\begin{table}[H]
\begin{center}
\begin{tabular}{p{5,3cm} p{10cm}}
\multicolumn{2}{c}{\textbf{Riesgo-23} } \\
\hline \hline
\textbf{Nombre del riesgo} & Avería en las cámaras \\
\hline
\textbf{Prevención de riesgos} & \begin{itemize}[-]
  \item Asegurar la cámara en un soporte para evitar que no se produzcan daños por caída.
  \item Hacer revisiones periódicas para evitar que se produzcan averías por su uso.
  \end{itemize} \\
\hline
\textbf{Plan de recuperación} &   \begin{itemize}[-]
  \item Reemplazar el equipo averiado por uno nuevo.
  \item Disponer de un servicio técnico capaz de arreglar este tipo de hardware.
  \end{itemize}\\
\hline
\end{tabular}
\caption{Plan de prevención y recuperación Riesgo-23}
\label{tab:Riesgo-23-Prev_Recup}
\end{center}
\end{table}

\begin{table}[H]
\begin{center}
\begin{tabular}{p{5,3cm} p{10cm}}
\multicolumn{2}{c}{\textbf{Riesgo-24} } \\
\hline \hline
\textbf{Nombre del riesgo} & Avería en la antena GPS \\
\hline
\textbf{Prevención de riesgos} & \begin{itemize}[-]
  \item Asegurar la antena GPS en un soporte para evitar que no se produzcan daños por caída.
  \item Hacer revisiones periódicas para evitar que se produzcan averías por su uso.
  \end{itemize} \\
\hline
\textbf{Plan de recuperación} &   \begin{itemize}[-]
  \item Reemplazar el equipo averiado por uno nuevo.
  \item Disponer de un servicio técnico capaz de arreglar este tipo de hardware.
  \end{itemize}\\
\hline
\end{tabular}
\caption{Plan de prevención y recuperación Riesgo-24}
\label{tab:Riesgo-24-Prev_Recup}
\end{center}
\end{table}

\begin{table}[H]
\begin{center}
\begin{tabular}{p{5,3cm} p{10cm}}
\multicolumn{2}{c}{\textbf{Riesgo-25} } \\
\hline \hline
\textbf{Nombre del riesgo} & Avería en el sensor de distancia  \\
\hline
\textbf{Prevención de riesgos} & \begin{itemize}[-]
  \item Asegurar el sensor en un soporte para evitar que no se produzcan daños por caída.
  \item Hacer revisiones periódicas para evitar que se produzcan averías por su uso.
  \end{itemize} \\
\hline
\textbf{Plan de recuperación} &   \begin{itemize}[-]
  \item Reemplazar el equipo averiado por uno nuevo.
  \item Disponer de un servicio técnico capaz de arreglar este tipo de hardware.
  \end{itemize}\\
\hline
\end{tabular}
\caption{Plan de prevención y recuperación Riesgo-25}
\label{tab:Riesgo-25-Prev_Recup}
\end{center}
\end{table}

\begin{table}[H]
\begin{center}
\begin{tabular}{p{5,3cm} p{10cm}}
\multicolumn{2}{c}{\textbf{Riesgo-26} } \\
\hline \hline
\textbf{Nombre del riesgo} & Avería en el hardware de control \\
\hline
\textbf{Prevención de riesgos} & \begin{itemize}[-]
  \item Asegurar el hardware en un soporte para evitar que no se produzcan daños por caída.
  \item Hacer revisiones periódicas para evitar que se produzcan averías por su uso.
  \end{itemize} \\
\hline
\textbf{Plan de recuperación} &   \begin{itemize}[-]
  \item Reemplazar el equipo averiado por uno nuevo.
  \item Disponer de un servicio técnico capaz de arreglar este tipo de hardware.
  \end{itemize}\\
\hline
\end{tabular}
\caption{Plan de prevención y recuperación Riesgo-26}
\label{tab:Riesgo-26-Prev_Recup}
\end{center}
\end{table}


\subsection{Monitorización y Control de Riesgos}
\par La monitorización y control de riesgos se llevará a cabo durante todo el desarrollo del proyecto, y en el Informe Quincenal de Seguimiento, deberá aparecer reflejado lo que se haya obtenido al respecto en las dos semanas que comprenda el IQS.

Se estudiará el impacto de los riesgos actuales, el coste de dichos riesgos y la posibilidad de añadir nuevos riesgos ya que a medida que avance el proyecto es probable que aparezcan nuevos riesgos. Además se comprobará si se está siguiendo correctamente lo referente a prevención de riesgos explicado en el apartado anterior.

En caso de aparecer nuevos riesgos deberá solicitarse su incorporación a este documento con una solicitud de cambio sobre el documento indicando que en el IQS se ha detectado un nuevo riesgo y desea añadirse.

\subsection{Planificación de la Gestión de Riesgos}
\par Como Jefe de Proyecto, Alberto García Hernández, establecerá las pautas adicionales que considere oportunas para la gestión de riesgos pero se toma como base indispensable las siguientes:
\begin{itemize}[-]
  \item Todos los miembros del proyecto deberán estar al tanto de los riesgos que pueden existir, así como las formas de prevenirlos y qué hacer en caso de detectarse un riesgo, lo cual ha sido establecido en los apartados anteriores.
  \item Se deberá cumplir con lo propuesto en el apartado 2.4 “Monitorización y Control de Riesgos” para poder comprobar que lo establecido en el punto anterior se esté cumpliendo.
\end{itemize}

\par El presupuesto asociado a riesgos ya fue calculado en el DCC y por tanto no se ahondará más en el coste del Plan de Gestión de Riesgos en este apartado.

\subsection{Impacto en el Coste del Sistema}
\par El Plan de Gestión de Riesgos tiene un impacto considerable en el coste del sistema, principalmente por el aumento de horas dedicadas al proyecto, sin embargo, al establecerse la revisión cada quince días, se estima que no debería implicar más de una hora semanal para comprobar que se está realizando lo establecido en lo relativo a prevención, salvo que se necesite tratar una incidencia.

El impacto económico solo remitirá en las horas extras que habría que dedicar en caso de producirse una incidencia, algo que ya habíamos tenido en cuenta a la hora del Cálculo de Costes del Proyecto.

En cuanto al impacto sobre la planificación, no es probable que sea necesario hacer una modificación en ésta, pero en caso de que así fuese, se indicaría en el IQS, donde se lleva el seguimiento de la planificación y el grado de avance del proyecto.
