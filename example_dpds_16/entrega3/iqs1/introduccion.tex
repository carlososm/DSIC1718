\section{Cálculo de la Priorización}

\subsection{Propósito y alcance del proyecto}
\par Con este documento se pretende realizar un seguimiento del estado actual del proyecto, para ello, se comparan en él las estimaciones que se realizaron en su momento con los datos reales obtenidos hasta la fecha. Gracias a esta comparación, será posible ajustar las estimaciones iniciales de modo que se obtenga una estimación más precisa y realista. Además, se observará si ha aparecido alguno de los riesgos definidos en el documento de Gestión de Calidad y si se ha actuado de acuerdo al plan de acción establecido.

Para cumplir con los objetivos definidos para este documento, se pondrán en común todas las tareas llevadas a cabo por los miembros del equipo. Cada uno de los miembros debe rellenar una hoja de imputación de horas, en la que refleje las horas dedicadas a cada uno de los productos generados hasta el momento, de forma que puedan compararse el tiempo real dedicado con el tiempo que se estimó en el documento de cálculo de costes.

Las tareas que se ven afectadas por este documento son aquellas cuya fecha de finalización estimada está entre el comienzo del proyecto y el día 05 de Abril del presente año, estas tareas son las siguientes: DCC, OFE, GCONF, PGCal.

\subsection{Acrónimos y definiciones}
\begin{itemize}[-]
  \item DAS: Documento de Análisis del Sistema.
  \item DCC: Documento de Cálculo de Costes.
  \item DDS: Documento de Diseño del Sistema.
  \item DHP: Documento de Histórico del Proyecto.
  \item DIS: Documento de Implantación del Sistema.
  \item EVS: Estudio de Viabilidad del Sistema.
  \item IAS: Implantación y Asimilación del Sistema.
  \item IQS: Informe Quincenal de Seguimiento.
  \item OFE: Oferta.
  \item GConf: Plan de Gestión de Configuración.
  \item PGCal: Plan de Gestión de Calidad.
  \item PER: Planificación y especificación de requisitos.
  \item  DCS: Documento de Construcción del Sistema.
\end{itemize}
