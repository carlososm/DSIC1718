\section{Introducción}

\subsection{Propósito del plan}
El Plan que a continuación se detalla, va dirigido tanto al personal desarrollador como al equipo de dirección. Con él se pretende dotar al proyecto de suficiente robustez a la hora de recopilar información acerca del estado del producto, así como a la hora de realizar un cambio. Los cambios son especialmente delicados en este, dado que existen elementos que requieren especial atención y cuidado a la hora de modificarlos.
Así pues se pretende documentar cada línea base y cada cambio realizado según lo indicado más abajo cuando se detallen las actividades de gestión de configuración.

\subsection{Alcance}
El presente plan de GCS se aplicará al proyecto realizado por Smart Software Solutions para CARSAFETY, el cual se corresponde con un sistema de gestión de la seguridad para vehículos, compuesto por cinco subsistemas: control del punto ciego, alerta de cambio de carril, alerta de velocidad, llamada automática de emergencia y sistema pre-colisión

\subsection{Definiciones y acrónimos}
A continuación aparecen las definiciones utilizadas en el presente plan de gestión de configuración.

\begin{description}[style=multiline, leftmargin=4cm]
  \item[Bibliotecas software:] es un repositorio de documentación y de colecciones de
  software que sirve como soporte para ayudar en el desarrollo de un proyecto.
  \item[Ciclo de vida:] es una secuencia estructurada y bien definida de las etapas necesarias para desarrollar un determinado producto software.
  Comité de Control de Cambios:] persona o conjunto de personas encargadas de supervisar y aprobar todos los cambios sugeridos.
  \item[Control de versiones:] se trata de la gestión de las diversas modificaciones realizados sobre los elementos del proyecto.
  \item[Elementos de configuración:] es la información creada como parte del proceso de un determinado proyecto.
  \item[Líneas base:] especificación o producto que ya se ha revisado formalmente, y sobre el que se ha llegado a un acuerdo. De esta manera, sirve como base para cualquier desarrollo posterior que se quiera realizar.
  \item[Petición de cambio:] solicitud que se presenta ante el CCC, que describe un cambio de cualquier tipo en el ciclo de vida natural del producto, o en aspectos relacionados.
  \item[Versión:] es el estado en el que se encuentra un proyecto en un momento determinado de su desarrollo.
  \item[Versión en desarrollo:] versión de un componente que todavía está sufriendo modificaciones y, por lo tanto, no está disponible para su uso.
  \item[Versión final:] versión de un componente que se encuentra disponible para el uso de usuarios finales.
\end{description}

A continuación, aparecen los acrónimos utilizados en el presente plan de gestión de configuración.
\begin{description}[style=multiline, leftmargin=2cm]
  \item[CCB:] Configuration Control Board. Comité de control de la configuración.
  \item[CI:] Configuration Item. Elemento bajo gestión de la configuración.
  \item[CM:] Configuration Management. Manejo de la gestión de la configuración.
  \item[SCM:] Software Configuration Management. Gestión de configuración del software.
  \item[SCMR:] SCM Responsible. Responsable del SCM.
  \item[SCMP:] Software Configuration Management Plan
  \item[SCR:] System/Software Change Request. Petición de cambio en el sistema/software.
  \item[CCC:] Comité de Control de Configuración.
  \item[CCR:] Responsable del CC.
  \item[EC:] Elemento de Configuración.
  \item[EVS:] Estudio de Viabilidad del Sistema
  \item[DAS:] Documento de Análisis del Sistema.
  \item[DCC:] Documento de Cálculo de Costes.
  \item[DDS:] Documento de Diseño del Sistema.
  \item[DHP:] Documento de Histórico del Proyecto.
  \item[DIS:] Documento de Implantación del Sistema.
  \item[IAS:] Implantación y Asimilación del Sistema.
  \item[IQS:] Informe Quincenal de Seguimiento.
  \item[OFE:] Oferta.
  \item[GConf:] Plan de Gestión de Configuración.
  \item[PGCal:] Plan de Gestión de Calidad.
  \item[PER:] Planificación y especificación de requisitos
  \item[DCS:] Documento de Construcción del Sistema.
\end{description}
