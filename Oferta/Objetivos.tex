% DOCUMENT

\section{Objetivos del sistema a desarrollar}
\subsection{Descripción del sistema}
\par El sistema a desarrollar consiste en desarrollar un sistema avanzado para mejorar la seguridad en los vehículos de CARSAFETY. Este proyecto opera tanto a nivel de hardware como a nivel de software, ya que será necesario instalar todos los componentes para detectar la información que necesite el sistema, así como los algoritmos necesarios para su correcta interpretación y comunicación con el usuario.
\par La propuesta abarca seis subsistemas diferentes:

\begin{itemize}[-]
\item \textbf{Control del punto ciego}
\par Existen ciertas zonas de visibilidad reducida desde los espejos retrovisores para el conductor. Cuando hay un vehículo en dicha zona y el conductor quiere hacer un cambio de carril, puede poner en peligro su propia seguridad y la de los demás conductores.
\par Es por eso que la empresa CARSAFETY ha solicitado que en el proyecto se incluya un control del punto ciego, el cual enviará una notificación al conductor cuando no sea seguro realizar la maniobra, es decir, cuando haya un vehículo en la zona del punto ciego. Para detectar al vehículo, se utilizarán dos sonar infrarrojos colocados en las esquinas traseras del vehículo, pudiendo detectarlos a una distancia máxima de 765 cm.
\item \textbf{Aviso cambio de carril}
\par En ocasiones el conductor no es consciente de la distancia entre el vehículo y la línea que delimita su carril, y puede llegar a invadirlo. No solo esto es un problema, sino que además el conductor ante esta situación puede reaccionar de forma brusca al volver a su carril y ponerse él mismo en peligro y a los demás conductores.
\par CARSAFETY ha tenido en cuenta esta situación y ha especificado que mediante la vibración del volante, el conductor se de cuenta si se está aproximando peligrosamente a la línea de delimitación del carril. Este sistema sólo se activará cuando se alcancen velocidades de autopista, y además el conductor podrá desactivar el sistema cuando active los intermitentes. Para desarrollar este subsistema se usará una cámara situada en la parte superior de la luna frontal del vehículo, para después mediante una FPGA poder procesar las imágenes.
\item \textbf{Alertas de velocidad}
\par Todas las vías tienen una velocidad máxima permitida, pero hay ocasiones en las que el conductor va a una velocidad superior a la permitida. Esto puede provocar que se produzcan accidentes de tráfico que involucren no solo al conductor del vehículo, sino también a los acompañantes y a otros vehículos que vayan circulando por la vía.
\par Gracias al sistema que se propondrá para CARSAFETY, se podrá determinar la velocidad de circulación máxima permitida por la vía, ya sea por el reconocimiento de las señales de tráfico o, en caso de que no haya, por GPS. De esta forma, si el conductor supera la velocidad máxima permitida se le avisará mediante una notificación sonora. Al igual que en el anterior subsistema, se utilizará una cámara situada en la parte superior de la luna frontal del vehículo, para después mediante una FPGA poder procesar las imágenes.
\item \textbf{Pérdida de atención}
\par Cuando el conductor de un vehículo está cansado, ya sea porque lleva muchas horas conduciendo o porque simplemente no tiene la energía necesaria para conducir, puede provocar una situación de peligro que involucre a otros conductores y a él mismo.
\par CARSAFETY ha solicitado poder reconocer cuándo un conductor está capacitado, o no, para conducir. Cada vez que se encienda el motor del coche, el sistema determinará la posición de los párpados del conductor y la presión que éste sobre el volante, y enviará una notificación sonora cuando sea peligroso iniciar la marcha. Si el conductor ya se encuentra conduciendo, también será posible detectar la posición de los párpados y la presión del volante cada cierto tiempo, y enviará una notificación sonora cuando esté empezando a quedarse dormido.
\par Además, el motor se irá deteniendo progresivamente y se encenderán las luces de emergencia. Cuando se haya detenido completamente, se activará de forma automática el freno de mano.
\par Para poder medir la presión que ejerce el conductor sobre el volante se utilizarán sensores para pedir la presión, la frecuencia y la posición de las manos. Además, con una cámara de reconocimiento facial se detectará la posición de los párpados del conductor.
\item \textbf{Llamada automática de emergencia}
\par Hay ocasiones en las que el conductor se encuentra en una situación de emergencia tras un accidente y no le es posible pedir ayuda.
\par Cuando ocurra esto, el sistema enviará al “Punto de respuesta de seguridad pública” un mensaje en formato europeo estándar con las coordenadas GPS del vehículo para que el conductor pueda recibir la ayuda necesaria.
\par Para este subsistema, nos ajustaremos al protocolo europeo para el formato del mensaje con las coordenadas GPS, por lo que solo será necesario un dispositivo capaz de transmitir la información en el momento del impacto.
\item \textbf{Alertas precolisión}
\par Los obstáculos que se encuentran en la misma trayectoria del vehículo pueden provocar graves accidentes. Algunos de esos obstáculos pueden ser detectados fácilmente por el conductor, pero hay otros que son más complicados de ver y no puede reaccionar correctamente.
\par Gracias a este sistema, el vehículo aplicará unas medidas preventivas para mejorar la seguridad del conductor cuando se detecte un obstáculo. Dichas medidas serán: reducir la velocidad, ajustar los cinturones de seguridad, cerrar las ventanillas y colocar los asientos en una posición óptima para que los airbag funcionen correctamente. Para este subsistema se utilizará un láser para la detección de obstáculos, los cuales podrán ser detectados a una distancia máxima de 215 cm.
\end{itemize}

\subsection{Ventajas del sistema}
\par Tal y como se ha detallado en el punto anterior, este sistema abarca muchos aspectos que mejorarán la seguridad en la conducción tanto del propio conductor del vehículo, como de otros agentes relacionados (viandantes, otros conductores que estén en la vía, etc). Las ventajas son las siguientes:
\begin{itemize}[-]
\item Reducir el número de accidentes de tráfico, ya que muchos de ellos se producen por causas evitables gracias a las nuevas tecnologías, las cuales se utilizarán para implementar el producto final.
\item Los organismos públicos de gestión de carreteras (DGT) ahorrará costes en  la reconstrucción de las carreteras debido a los accidentes de tráfico.
\item Las empresas aseguradoras ahorrarán en las reparaciones de los vehículos involucrados en los accidentes de tráfico.
\item Las empresas aseguradoras que proporcionan seguros de vida reducirán costes al haber menos accidentes de tráfico.
\end{itemize}
