\section{Requisitos iniciales de usuario}\label{sec:user_req}
\par Para el desarrollo del subsistema se han de tener en cuenta los requisitos indicados por el cliente. A lo largo de este apartado, estos serán identificados y explicados de manera textual, como resultado de las entrevistas realizadas con el cliente. Más adelante, en el análisis del sistema (Sección \ref{sec:analisis}), se verán los requisitos analizados y detallados.

\par De esta forma, los requisitos aquí esbozados deberán ser cubiertos por todos los requisitos detallados en el apartado de análisis.

\par Tras las diferentes reuniones mantenidas con el cliente, se han distinguido los siguientes requisitos de usuario:


\begin{itemize}[-]
    \item El subsistema desarrollará un portal corporativo de gestión inmobiliaria de otros espacios.
    \item El subsistema deberá ser capaz de dar de alta otros espacios en el portal corporativo.
    \item El subsistema deberá ser capaz de dar de baja otros espacios en el portal corporativo.
    \item El subsistema deberá ser capaz de dar de modificar información de otros espacios en el portal corporativo.
    \item La información introducida deberá poder ser vista por los clientes.
    \item La información introducida sobre otros espacios deberá ser, al menos, el tipo de espacio, el precio de venta, el  precio de alquiler, la foto de la propiedad, la localización y el tamaño del espacio.
    \item Antes de ser publicada la información, el coordinador de área deberá revisarla y dar el visto bueno a la misma,
    \item Se deberá incluir la fecha de publicación, la fecha en el que el espacio puede comenzar a ser visible por los clientes y la fecha en la que dejará de serlo.
    \item El subsistema deberá tener un control de los usuarios y los grupos de usuarios que pueden acceder a la información y de los permisos que tienen estos sobre la información.
    \item El subsistema deberá permitir que los comerciales y los coordinadores de área compartan información entre ellos a través del portal corporativo sin que los usuarios que no tengan permiso puedan verlos.
\end{itemize}
