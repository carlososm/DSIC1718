\chapter{Documentación de Entrada}\label{sec:initialdocs}
\par A continuación se mostrará la documentación inicialmente proporcionada por el cliente a partir de la cual se ha generado el documento y el proyecto. Después de mostrar el texto original se hará un extracto de lo que RSPlusAgency ha entendido que quiere el cliente.
\subsection{Extracto de la documentación original}
\par La documentación original fue entregada en un archivo pdf por el cliente José Luis López Cuadrado a través de Aula Global.


\begin{center}
    \begin{minipage}{0.9\linewidth}
        \vspace{5pt}%margen superior de minipage
        {\small
        Un agencia dedicada a la compra, venta y alquiler de viviendas quiere modernizar sus
        sistemas de cara a gestionar la oferta y la demanda de sus productos.
        La empresa dispone de un conjunto de comerciales que contactan con propietarios de
        pisos, chalets, locales, solares y otros espacios (como por ejemplo naves industriales)
        que pueden estar interesados en la venta o alquiler de los mismos. Cuando el
        propietario decide poner su activo en manos de la agencia, se publica un anuncio en el
        que se incluye una foto de la propiedad y los datos correspondientes a la misma. Por
        ejemplo para pisos se incluye número de habitaciones, metros totales disponibles, si
        tiene garaje, trastero, etc. Finalmente todos los elementos ofertados tienen un precio
        asociado.
        La agencia tiene comerciales especializados en pisos, locales comerciales, plazas de
        aparcamiento, chalets, solares y otros espacios (naves, edificios completos...). Antes
        de publicar una oferta el coordinador de cada área comercial debe dar el visto bueno a
        la misma.
        La Agencia nos ha encargado un sistema de información corporativo que les permita
        por un lado gestionar la información de sus productos en el sistema por parte de los
        comerciales, y por otro lado facilitar el acceso por parte de los clientes
        Existen seis áreas: alquiler de locales, alquiler de pisos, venta de pisos, venta de
        locales, venta de solares y otros activos. Cuando un editor recibe la información de una
        actividad, tiene que darla de alta en el sistema para que sea visible desde el portal de
        información, indicando la fecha en que la información puede empezar a ser visible y la
        fecha en la que debe dejar de serlo.
        Cuando el comercial a introducido la información de la propiedad a gestionar, el
        coordinador de área debe dar el visto bueno. Cuando el coordinador de área da el
        visto bueno, la información de la actividad pasa a estar visible hasta la fecha que se
        haya definido como límite para la publicación.
        Se pide personalizar un portal Liferay para soportar los procesos de la organización,
        definiendo los roles y la funcionalidad correspondiente, junto con los elementos
        necesarios para que toda la información sea accesible desde la web.

        ...
        ...

        Subsistemas a desarrollar
        Como se ha indicado, existen seis áreas diferentes de actividad dentro de la
        organización, y cada una de ellas necesitará su subsistema específico: alquiler de
        locales, alquiler de pisos, venta de pisos, venta de locales, venta de solares y otros
        activos. Cada una de las áreas define un subsistema.
        }
        \begin{flushright}
            (\citeauthor{WEB:Enunciado}, \citeyearNP{WEB:Enunciado}: 10)
        \end{flushright}
        \vspace{5pt}%margen inferior de la minipage
    \end{minipage}
\end{center}



\subsection{Análisis de la documentación original}
\par Lo que hemos entendido por parte del cliente, es que desarrollemos lo siguiente:
\begin{itemize}
	\item \textbf{Un portal web} mediante la tecnología LifeRay. Este portal debe permitir \textit{gestionar la información de sus productos en el sistema por parte de los
comerciales, y por otro lado facilitar el acceso por parte de los clientes}.
	\item \textbf{Diseñar los roles} que administren dicho portal.
	\item \textbf{Desarrollar la funcionalidad} necesaria para una de las áreas de dicha agencia, que pueden ser:
	\begin{itemize}
	 	\item Alquiler de locales
	 	\item Alquiler de pisos
	 	\item Venta de pisos
	 	\item Venta de locales
	 	\item Venta de solares
	 	\item Otros activos
	 \end{itemize}
	 \item \textbf{La documentación} necesaria para llevar a cabo el proyecto.
	 \item \textbf{Desplegar} la aplicación web del portal Liferay ya configurado y listo para ser usado.
\end{itemize}
