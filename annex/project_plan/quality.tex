\section{Plan de Gestión de la Calidad} \label{sec:calidad}
\subsectionsection{Introducción}
\par En este documento se recogen esas tareas a realizar para garantizar la calidad del proyecto a desarrollar. 

\subsubsection{Plan de Calidad}
\par A continuación se expondrán las tareas que se realizarán en el Plan de Calidad para comprobar que el proyecto cumple los criterios de calidad necesarios estimados por el cliente.

Las revisiones se realizarán de forma periódica a medida que se completen fases del proyecto hasta completar el producto.

En los siguientes puntos se detallan las revisiones específicas a realizar y por cada una de las revisiones se generará un informe de auditoría, recogiendo la aprobación o rechazo del producto por parte del comité que haga la revisión, indicando las causas de la decisión tomada.

\subsubsection{Revisiones de los Requisitos}
\par El responsable de calidad validará los requisitos una vez se hayan especificado de forma estructurada, siguiendo lo establecido por el Plan de Calidad. Estas revisiones comprobarán lo siguiente:
\begin{itemize}[-]
  \item Identificación de los requisitos de usuario.
  \item Cada requisito describe la funcionalidad que le corresponde.
  \item Correspondencia entre los requisitos generados y los requisitos obtenidos del usuario.
  \item Descripción de los requisitos en un lenguaje claro y no ambigüo.
  \item Se realizará una matriz de trazabilidad para comprobar que todos los requisitos de usuario tienen asociado al menos un requisito de software.
\end{itemize}
\par Esta revisión la llevará a cabo el jefe de proyecto una vez se haya generado todos los requisitos.

\subsubsection{Revisiones de Consistencia}
\par El responsable de calidad se encargará de realizar revisiones de consistencia entre los productos generados en el proyecto, por lo que se comprobarán los siguientes aspectos del software desarrollado:

\begin{itemize}
\item \textbf{Funcionalidad}: el software debe proveer las funciones que cumplen con las necesidades definidas por los requisitos.
\item \textbf{Fiabilidad}: el software debe mantener un cierto nivel de rendimiento pactado entre RSPlusAgency y el cliente.
\item \textbf{Usabilidad}:el producto software debe ser práctico, eficaz y con una curva de aprendizaje poco pronunciada para facilitar su uso al usuario final.
\item \textbf{Eficiencia}:el software debe ofrecer un cierto rendimiento respecto a la cantidad de recursos utilizados en un entorno declarado por el cliente.
\item \textbf{Portabilidad}:el software debe ser capaz de ser trasladado de un entorno a otro.
\end{itemize}

\par Como resultado, al igual que en la revisión de requisitos, se generará un informe de auditoría recogiendo la aprobación o rechazo del producto en función de los aspectos mencionados.

\subsubsection{Monitorización de Riesgos}
\par Durante el desarrollo del proyecto se llevará a cabo una monitorización de los riesgos, tanto los que se detectasen inicialmente como los que vayan surgiendo, bien por sucesivas revisiones o bien por causas externas. Se estudiará el coste de dichos riesgos y se añadirán a la sección de Análisis de Riesgos.