\section{Plan de Gestión de la Calidad}

\par A continuación se describirán los distintos métodos y actividades que se utilizarán a lo largo del proyecto para asegurar la calidad de éste. Entre dichos métodos se encuentran:
\begin{itemize}[-]
  \item Revisiones periódicas
  \item Reuniones de Seguimiento
  \item Pruebas
\end{itemize}

\subsection{Revisiones Periódicas}
\par Las revisiones se realizarán periódicamente cada semana los lunes de 10AM a 12AM a menos que el cliente lo modifique por alguna causa justificada. Esta hora puede ser cambiada más adelante debido a que tanto el cliente como RSPlus lleguen a un acuerdo.
\par Dichas revisiones incluyen, entre otros:
 \begin{itemize}[-]
  \item Documentación: \par El cliente habrá revisado previamente a la reunión los documentos facilitados por RSPlus y durante dicha reunión discutirá los puntos a revisar que vea convenientes. En el caso de que el cliente no haya encontrado ningún fallo o cambio a realizar pero RSPlus sí que lo haya detectado, se informará durante dicha reunión.
  \item Requisitos y casos de uso: \par Conforme se desarrolle el proyecto es posible que ciertos requisitos vayan cambiando de acuerdo al desarrollo del mercado o a ciertas especificaciones que no se puedan cumplir o al contrario, que se deban cumplir y no se detectasen en la elaboración de los requisitos.
  \item Pruebas: \par El cliente repasará con RSPlus las pruebas a realizar en el proyecto para que el software cumpla con los requisitos exigidos en el proyecto por parte del cliente. Estas pruebas estarán asesoradas por el equipo de RSPlus para ayudar al cliente en el caso de que no cuente con un equipo especializado en las tecnologías del proyecto.
\end{itemize}

\subsection{Reuniones de Seguimiento}
\par Las reuniones de seguimiento se realizarán periódicamente cada semana los jueves de 11AM a 1PM a menos que el cliente lo modifique por alguna causa justificada. Esta hora puede ser cambiada más adelante debido a que tanto el cliente como RSPlus lleguen a un acuerdo.

\par En las reuniones se discutirán temas acerca de:
\begin{itemize}[-]
  \item El proceso de desarrollo: \par Donde se comprobará el avance semanal por parte del equipo de desarrollo en el proyecto, así como las directivas y metodologías que se usan de acorde a lo especificado por el proyecto.
  \item Hitos: \par El cliente estudiará junto a RSPlus el estado de completitud de los hitos y si siguen siendo viables para las fechas inicialmente señaladas.
  \item Pruebas: \par Donde el cliente podrá comprobar qué pruebas han sido exitosas y cuáles no lo han sido.
\end{itemize}

\par Con estas reuniones se quiere conseguir una mayor granularidad al discernir qué beneficia o no al proyecto y si se están cumpliendo las pautas pactadas en el proyecto.

\subsection{Pruebas}
\par Durante el desarrollo se irá realizando pruebas que verifiquen que todos los requisitos, tanto funcionales como no funcionales, sean cumplidos acorde a lo especificado en el proyecto. Estás pruebas serán establecidas inicialmente por RSPlus, y se discutirán tanto en las Revisiones Periódicas con el cliente como en las Reuniones de Seguimiento, de forma que las pruebas garanticen los requisitos.
