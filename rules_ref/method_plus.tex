\subsection{Métodos, Herramientas, Modelos, Métricas y Prototipos}
\subsubsection{Métodos y Herramientas}
\par La elaboración de este documento, así como la realización del proyecto se llevar a cabo siguiendo el ISO/IEEE 16326:2009. Las herramientas utilizadas para la elaboración del documento son:
\begin{itemize}
 	\item LaTex: para programar los documentos pdf.
 	\item Sublime Text 3: utilizado para programar en LaTex.
 	\item Atom: utilizado para programar en LaTex.
 	\item MS Project: utilizado para la realización de la planificación temporal.
 	\item Libreoffice 5: utizado para la elaboración de tablas de cálculo.
 	\item Draw.io: utilizado para la creación de gráficos.
 	\item Linux: Debian 9 y Ubuntu 16.10
 	\item macOS
 	\item Windows 10
 \end{itemize} 
 \par Para el desarrollo del portal web se utiliza:
 \begin{itemize}
 	\item Liferay EE 7: aplicación java de portales web.
 	\item Java 8 EE: lenguaje de programación.
 	\item Navegadores web como Firefox, Chrome o Safari
 \end{itemize}

 \par Como metodología además se ha escogido la metodología de Craig Larman, que se compone de tres fases:
 \begin{itemize}
 	\item \textbf{Planificación y especificación de requisitos:} esta etapa consiste en definir todos los requisitos del sistema, además de los casos de uso.
 	\item \textbf{Construcción:} es una etapa iterativa que va refinando el proyecto con cada nueva iteración, lo que permite adaptar el proyecto a los improvistos que vayan surgiendo.
 	\item \textbf{Instalación:} una vez terminadas las etapas de construcción se procede a entregar e instalar la versión final.
 \end{itemize}


\subsubsection{Modelos, Métricas y Prototipos}
\par Como métrica utilizaremos las pautas del documento de Revisión Sistemática de Métricas de Diseño Orientado a Objetos de Juan José Olmedilla, utilizado en proyectos anteriores con muy buenos resultados, por lo que se seguirán las siguientes pautas:
\begin{itemize}
	\item \textbf{Funcionalidad:} comprobación de la capacidad del software que asegura que provee las necesidades descritas por los requisitos.
	\item \textbf{Fiabilidad:} el software deberá mantener un nivel de rendimiento bajo unas condiciones de uso definidas.
	\item \textbf{Usabilidad:} el producto debe ser fácilmente entendido y de rápido aprendizaje.
	\item \textbf{Eficiencia:} el software debe proporcionar un rendimiento adecuado a la cantidad de recursos que utilice.
	\item \textbf{Mantenibilidad:} el software debe permitir modificaciones posteriores a cuando sea terminado.
	\item \textbf{Portabilidad:} el software debe ser capaz de ser trasladado a los entornos que designe el cliente sin complicaciones adicionales.
\end{itemize}

\par Siguiendo ese modelo de desarrollo creemos que se conseguirá un nivel óptimo de calidad y nos permitirá realizar el proyecto sin altercados por parte del equipo de desarrollo.