\subsection{Mecanismos de control de calidad aplicados durante la redacción del proyecto}
\par Los mecanismos de control aplicados durante la redacción del proyecto son los siguientes:
\begin{itemize}
	\item \textbf{Comunicación directa con el cliente} para aclarar cualquier duda o punto ambigüo que pudiese existir, para una posterior corrección.
	\item \textbf{Discusión grupal} de los distintos puntos del documento, repartiendo el escrutinio entre los integrantes del equipo de tal forma que el integrante que realizase la tarea A revise la tarea B y viceversa, asegurando que todos los puntos del documento han sido revisados por al menos una persona que no los creó originalmente. Esto asegura que cada tarea está correctamente realizada y revisada.
	\item \textbf{Revisión del documento con ejemplos oficiales} proporcionados por la universidad Carlos III de Madrid u organizaciones como IBM o IEEE.
	\item \textbf{Revisión del documento por parte del cliente} corrigiendo los errores encontrados por éste y entregándolo en la siguiente iteración corregido.
\end{itemize}