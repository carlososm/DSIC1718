\section{Valoración de las alternativas}
\par Una vez presentadas las distintas alternativas a cada uno de los subsistemas, se debe proceder a una valoración de las mismas tanto en el ámbito económico como en el contexto de los riesgos derivados de las mismas. Así, mediante esta valoración y la búsqueda de la disminución de los riesgos de cada una de las alternativas, resultará más sencilla la selección final de una de las alternativas como solución final.

\par Por tanto, se hará una valoración de los riesgos de las alternativas en cada uno de los susbsistemas, así como un análisis de la viabilidad económica de las mismas. Para este segundo caso, se ha de tener en cuenta la inexistencia de restricciones económicas por parte del cliente.


\subsection{Gestión de riesgo}
\par Valorando en primer lugar la única alternativa propuesta para el subsistema de llamada de emergencia, el mayor de los riesgos es que el dispositivo utilizado resulte dañado en una colisión o accidente. Para ello, la mejor solución es ubicarlo en un emplazamiento que sufra pocos daños (cubículo del conductor) y protegido físicamente.

\par En segundo lugar, en cuanto a la única alternativa propuesta para el subsistema de alerta en caso de pérdida de atención es que las mediciones tomadas son subjetivas, aunque la prevención de este problema es compleja y está completamente ligada al software y los algoritmos utilizados.

\par A continuación, se valorará cada una de las alternativas propuestas.

\subsubsection{Alternativa I}
Analizando la alternativa propuesta para cada uno de los subsistemas, podemos detectar un primer riesgo global, subyacente al hecho de que tres de los subsistemas se controlen mediante el mismo hardware y software. Este hecho puede provocar que, ante una caída parcial de la solución aplicada, tres de las funciones del sistema dejen de funcionar. Este es el problema común a todos los sistemas centralizados.

Los mismos riesgos se pueden observar en el sistema de comunicación central, también centralizado. Sin embargo, estos riesgos derivados de la centralización se pueden reducir proponiendo medidas que aumenten la fiabilidad del sistema y que monitoricen las posibles caídas del sistema, para prevenirlas y advertir al usuario antes de las mismas.


Finalmente, en cuanto al sistema central, además de los riesgos ya mencionados, cabe destacar lo prematuro del sistema operativo. Es decir, el usar la versión de un sistema operativo muy actualizada siempre tiene las ventajas derivadas de la labor de mejora del sistema operativo que se produce con cada versión nueva, pero supone el riesgo de nuevos errores no encontrados aún y de posibles incompatibilidades con software más antiguo.

\subsubsection{Alternativa II}
\par En cuanto a la alternativa dos se refiere, y de igual forma que se ha visto en la alternativa uno, el mayor de los riesgos que se observa es la centralización del sistema. En este caso, los subsistemas de alerta de velocidad, cambio de carril, alerta de precolisión y control del punto ciego dependen de un mismo hardware y software, de nuevo centralizados.

\par Además de este riesgo central, el uso de un protocolo de comunicación inalámbrico supone diversos riesgos a pesar de reducir costes en cableado y la dificultad de realizarlo. Estos riesgos son derivados de los sistemas inalámbricos de redes. Una comunicación basada en este protocolo es menos fiable, por la posibilidad de pérdida de información entre el emisor y el receptor. Además, la velocidad de transmisión es algo menor.

\par Para contrarrestar estos riesgos, cabe decir que diversas implementaciones del protocolo ZigBee lo hacen muy resistente frente a pérdidas, y por tanto, más seguro en las transmisiones. Además, aplicando de manera completa el protocolo mediante un buen software, la prevención de este tipo de problemas está asegurada.

\par Por otro lado, otro de los riesgos de esta alternativa se debe a sus limitaciones. El hecho de necesitar instalar el software por encima del chasis, aumentando la altura del vehículo, supone un riesgo físico. Este aumento de la posibilidad de rotura o avería, sumado al riesgo ya explicado de la suma de susbsistemas regulados por el mismo hardware, hace que el riesgo aumente exponencialmente. Para reducirlo, se debe prestar especial atención a la protección física de la cámara ubicada en la parte superior, además de favorecer el conocimiento por parte del usuario de la mayor de altura de su vehículo.

\subsubsection{Alternativa III}
\par Pasando a la alternativa tres, se debe tener en cuenta la reducción del riesgo de centralización con respecto a las dos alternativas anteriores. En este caso, un mismo mecanismo sólo regula el funcionamiento de dos subsistemas, reduciendo el riesgo antes citado.

\par No obstante, en cuanto a la alerta de velocidad se refiere, y más en concreto al respecto de los mapas, existe el riesgo de que Nokia deje de actualizar los citados mapas, dejando de dar soporte a la aplicación y surgiendo pues la necesidad de sustituirlos. Para disminuir este riesgo, se debe hacer el sistema adaptable a otro tipo de mapas.




\subsection{Viabilidad económica}

\par Lo primero que debemos destacar en este apartado es la ausencia de restricciones económicas por parte del cliente, lo que conlleva la viabilidad económica teórica de todas las alternativas. Es por ello por lo que en este apartado nos focalizaremos en una comparativa económica de cada una de las alternativas propuestas.

\par  En cuanto a la primera alternativa, el coste total asciende a 3404,3 euros (sin tener en cuenta el coste de la alternativa propuesta para la llamada de emergencia y la pérdida de atención, pues es una alternativa común). La segunda alternativa tiene un coste de 2656,19 euros, con las mismas salvedades que en el caso anterior. Finalmente, la tercera alternativa tiene un coste de 1866,46 euros.

\par  Tras una primera aproximación, se realiza el análisis económico para cada uno de los subsistemas. Para ello, cuando una misma solución o alternativa sea utilizada para varios subsistemas, se hará la media de coste para cada uno de los subsistemas.

\par Así, en la primera alternativa, podemos observar un coste de 39,90 euros para el sistema de comunicación; un coste de 1682,20 euros para el sistema de punto ciego; y un coste de  1682,20 para los sistemas de cambio de carril, alerta de velocidad y prevención de la pre-colisión (una media de 560,33 para cada uno de los subsistemas).

\par  En la segunda alternativa, el sistema general tiene un coste de 1881,19 euros, y los subsistemas de control de velocidad, alerta de pre-colisión, control del punto ciego y cambio de carril tienen un coste de 775 euros (193,75 de media).

\par Por su parte, el sistema general de comunicación de la tercera alternativa tiene un coste de 379,3 euros; el subsistema de control de punto ciego, 42,25 euros; el subsistema conjunto de cambio de carril y alerta de velocidad, de 900,06 euros (405,03 de media cada uno); y el subsistema de alerta de pre-colisión, de 544,85 euros.

\par Como se puede ver, los costes totales de las tres alternativas son similares, aunque su distribución de costes entre los distintos subsistemas son distintos. En el caso de la primera alternativa, el coste del sistema central es muy barato, debido a que el procesamiento de imágenes de hace en las FPGA de cada uno de los subsistemas. Es ahí donde se encarece la solución. Por otra parte, en la segunda alternativa, el integrar en un único hardware y software la solución propuesta, este se encarece mucho, pero el coste medio de cada subsistema es económicamente más barato. Por último, podemos ver que en la tercera alternativa el coste está muy equilibrado entre los distintos subsistemas.
