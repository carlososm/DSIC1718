\subsection{Descripción del entorno actual}
\par En la actualidad, hay muchas empresas virtuales que administran inmuebles como:
\begin{itemize}
	\item Idealista
	\item Fotocasa
	\item Vivienda2
\end{itemize}

\par Todas estas empresas se han hecho un hueco en la comunidad online gracias a la alta usabilidad de su interfaz y a la facilidad para publicar secciones con inmuebles, sin embargo (y como se verá más adelante) todas estas páginas tienen algunos factores que no las hacen perfectas.

\par Además de lo anterior, hay varias opciones para crear portales web además de liferay, como:
\begin{itemize}
	\item Weblogic Portal
	\item Jira
	\item Alfresco
\end{itemize}

\par Todas esas soluciones son muy válidas para el proyecto que se va a realizar, sin embargo la que más flexibilidad y facilidad de desarrollo ofrece es LifeRay.

\par Actualmente, la empresa asesorada no cuenta con un Portal Corporativo integrado, sino que dispone de una página web pública a la que los futuros clientes pueden acceder para consultar información, pero no pueden realizar compras ni alquileres online. Así mismo, para la gestión corporativa, cuenta con una base de datos única a la que acceden para ver la situación de los inmuebles a través de una aplicación de escritorio (desacoplada de la página web).

\par Lo que se pretende desarrollar en este proyecto debe tener las siguientes características:
\begin{itemize}
	\item \textbf{Un portal web} desde el cual el cliente pueda administrar la empresa y publicar las ventas y alquileres de activos como:
	\begin{itemize}
	 	\item Alquiler de locales
	 	\item Alquiler de pisos
	 	\item Venta de pisos
	 	\item Venta de locales
	 	\item Venta de solares
	 	\item Otros activos
	 \end{itemize}
	 \item \textbf{Desplegar} la aplicación web del portal Liferay ya configurado y listo para ser usado.
\end{itemize}
\par Dichas características no son novedosas o singulares, sino que están bien establecidas en el mercado digital por lo que el factor diferenciador será que en este caso la empresa y no los usuarios, será quien gestione los activos.
