\section{Cálculo de la Priorización}\label{sec:prioCalc}

\par Para priorizar los casos de uso, asignaremos a cada una de las características un valor entre 1 (impacto insignificante) y 5 (impacto muy significativo).

\begin{table}[H]
\begin{center}
\begin{tabular}{  c  c  c  c  c  c  c  c  }
  	Factor & \textbf{Factor 1} & \textbf{Factor 2} & \textbf{Factor 3} & \textbf{Factor 4} & \textbf{Factor 5} & \textbf{Factor 6} & Total \\ \hline \hline
  	CDU-01 & 2 & 1 & 2 & 2 & 1 & 1 & 1.75 \\ \hline
  	CDU-02 & 2 & 1 & 2 & 2 & 2 & 1 & 1.85 \\ \hline
  	CDU-03 & 3 & 1 & 4 & 4 & 3 & 3 & 3.35 \\ \hline
  	CDU-04 & 3 & 1 & 4 & 4 & 3 & 5 & 3.55 \\ \hline
  	CDU-05 & 4 & 3 & 4 & 5 & 2 & 2 & 3.85 \\ \hline
  	CDU-06 & 3 & 2 & 2 & 1 & 3 & 3 & 2.20 \\ \hline
  	CDU-07 & 1 & 2 & 1 & 1 & 1 & 1 & 1.05 \\ \hline
  	CDU-08 & 2 & 1 & 4 & 5 & 3 & 3 & 3.35 \\ \hline
  	CDU-09 & 4 & 3 & 4 & 5 & 4 & 5 & 4.35 \\ \hline
  	CDU-10 & 2 & 2 & 2 & 2 & 3 & 2 & 2.10 \\ \hline
  	CDU-11 & 3 & 3 & 3 & 4 & 4 & 4 & 3.50 \\ \hline
  	CDU-12 & 4 & 2 & 4 & 3 & 5 & 4 & 3.70 \\ \hline
  	CDU-13 & 3 & 1 & 1 & 2 & 2 & 3 & 2.20 \\ \hline
\end{tabular}
\caption{Ponderación de las caracteristicas para los casos de uso}
\label{tab:priorizacion}
\end{center}
\end{table}

\par Para priorizar los casos de uso, hemos tenido en cuenta las dependencias existentes entre ellos, por lo que hemos tenido que reorganizarlos para obtener un orden coherente y realizable. La siguiente tabla incluye la priorización (ordenados de mayor a menor) teniendo en cuenta la dependencia entre los casos de uso. La organización se ha dividido también en tres iteraciones (Craig-Larman), lo que permite que en cada ciclo se implementen funcionalidades individuales y completas. La división de los casos de uso por iteraciones han quedado de la siguiente manera:

\begin{table}[H]
\begin{center}
\begin{tabular}{p{2cm} p{2cm} p{8cm}}
  \multicolumn{3}{c}{\textbf{Primera Iteración} } \\ \hline \hline
  \textbf{Caso de Uso} & \textbf{Ponderación} & \textbf{Comentarios} \\ \hline \hline
  CDU-09 & 4,35 & Tiene una dependencia \textit{include} con CDU-13, por lo que CDU-13 tiene que a la vez orealizarse antes. \\ \hline
  CDU-05 & 3,85 & \\ \hline
  CDU-12 & 3,70 & \\ \hline
  CDU-11 & 3,50 & \\ \hline
  CDU-13 & 2,20 & \\ \hline
\end{tabular}
\caption{Casos de uso que se realizarán en la primera iteración}
\label{tab:iteracion1}
\end{center}
\end{table}

\begin{table}[H]
\begin{center}
\begin{tabular}{p{2cm} p{2cm} p{8cm}}
  \multicolumn{3}{c}{\textbf{Segunda Iteración} } \\ \hline \hline
  \textbf{Caso de Uso} & \textbf{Ponderación} & \textbf{Comentarios} \\ \hline \hline
  CDU-04 & 3,55 & Tiene una dependencia \textit{include} con CDU-03, por lo que tiene que realizarse antes o a la vez que CDU-03. \\ \hline
  CDU-03 & 3,35 & \\ \hline
  CDU-08 & 3,35 & \\ \hline
  CDU-06 & 2,20 & \\ \hline
\end{tabular}
\caption{Casos de uso que se realizarán en la segunda iteración}
\label{tab:iteracion2}
\end{center}
\end{table}

\begin{table}[H]
\begin{center}
\begin{tabular}{p{2cm} p{2cm} p{8cm}}
  \multicolumn{3}{c}{\textbf{Tercera Iteración} } \\ \hline \hline
  \textbf{Caso de Uso} & \textbf{Ponderación} & \textbf{Comentarios} \\ \hline \hline
  CDU-10 & 2,1 & \\ \hline
  CDU-02 & 1,85 & Tiene una dependencia \textit{include} con CDU-01, por lo que tiene que realizarse antes o a la vez que CDU-01. \\ \hline
  CDU-01 & 1,75 & \\ \hline
  CDU-07 & 1,05 & \\ \hline
\end{tabular}
\caption{Casos de uso que se realizarán en la tercera iteración}
\label{tab:iteracion3}
\end{center}
\end{table}
