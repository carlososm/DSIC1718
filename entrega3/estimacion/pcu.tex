\section{Puntos de caso de uso sin ajustar}

\par Para comenzar con la estimación hay que tener en cuenta las dos fórmulas utilizadas en el proceso. La ecuación \ref{eq:puntoscasosusosinajustar} se sustituye en la ecuación \ref{eq:puntoscasosuso}, por lo que en primer lugar se procedera al cálculo de $UUCP$.
\begin{equation} \label{eq:puntoscasosuso}
  UCP = UUCP \times TCF \times EF
\end{equation}

\text Dónde:
\begin{itemize}[-]
  \item $UCP =$ Puntos de casos de uso
  \item $UUCP =$ Puntos de casos de uso sin ajustar
  \item $TCF =$ Factor de complejidad técnica
  \item $EF =$ Factor de entorno
\end{itemize}


\begin{equation}\label{eq:puntoscasosusosinajustar}
  UUCP = UAW + UUCW
\end{equation}
\text Dónde:
\begin{itemize}[-]
  \item $UUCP =$ Puntos de casos de uso sin ajustar
  \item $UAW =$ Peso de los actores sin ajustar
  \item $UUCW =$ Peso de los casos de uso sin ajustar
\end{itemize}

\par Para el cálculo de $UAW$ en la ecuación ref{eq:puntoscasosusosinajustar} se tienen que dividir los actores en simples, promedio y complejos dependiendo de la dificultad que representen. En la tabla \ref{tab:uaw} se dividen todos los actores del sistema, explicados en la sección \ref{sec:actors}, entre los grupos de dificultad.

\begin{table}[H]
\begin{center}
\begin{tabular}{l l p{5cm} c}
\textbf{ACTOR} & \textbf{DIFICULTAD} & \textbf{JUSTIFICACIÓN} & \textbf{PONDERACIÓN}\\ \hline \hline
Objeto externo & Medio & Se conecta con el sistema a través de un \textit{LIDAR} como se expuso en la sección \ref{sec:solucion} & 2  \\
Reloj & Medio & Los casos de uso que tienen como actor el reloj significa que se inician a partir de un suceso interno del sistema. & 2\\
Conductor & Complejo & Se trata de una persona que interactúa con el sistema por medio de una interfaz. & 3\\
Señales & Medio & Interactúa con el sistema a través de las cámaras y software de \textit{machine learning}.& 2\\
GPS & Fácil & Se conecta con el sistema a través de una API de programación. & 1\\
Líneas de carril & Medio & Interactúa con el sistema a través de cámaras y software de \textit{machine learning}. & 2 \\ \hline
\textbf{TOTAL} &  & & \textbf{12}\\ \hline \hline
\end{tabular}
\caption{Factor de peso de los actores sin ajustar.}
\label{tab:uaw}
\end{center}
\end{table}

\par El siguiente paso consiste en calcular $UUCW$ en la ecuación \ref{eq:puntoscasosusosinajustar}. En este cálculo entra en juego el concepto de transacción, que para este caso se ha basado en el número de requisitos con los que contaba cada caso de uso, se puede comprobar en la matriz de trazabilidad de la sección \ref{sec:matrizTraz}. En la tabla \ref{tab:uucw} se recoge el número de casos de uso que hay para cada nivel de complejidad, siempre basándose en la matriz de trazabilidad.

\begin{table}[h]
\begin{center}
\begin{tabular}{l c c c c}
\textbf{DIFICULTAD} & \textbf{\# TRANSACCIONES} & \textbf{\#CASO DE USO} & \textbf{PONDERACIÓN} & \textbf{TOTAL}\\ \hline \hline
Fácil & Entre 1 y 3 requisitos & 5 & 5 & 25  \\
Medio & Entre 4 y 7 requisitos & 5 & 10 & 50\\
Complejo & Más de 8 requisitos & 3 & 15 & 45\\ \hline
\textbf{TOTAL} & & & & \textbf{120}\\ \hline \hline
\end{tabular}
\caption{Factor de peso de los casos de uso sin ajustar.}
\label{tab:uucw}
\end{center}
\end{table}

\par De esta manera, sumando $UUCW$ y $UAW$ se obtiene los puntos de caso de uso sin ajustar, como se ve en la ecuación \ref{eq:resultpuntoscasosusosinajustar}.

\begin{equation} \label{eq:resultpuntoscasosusosinajustar}
  UUCP = 120 + 12 = 132
\end{equation}

\par A continuación, se pasa al cálculo de $UCP$, ecuación \ref{eq:puntoscasosusosinajustar}. El primer paso es calcular el factor de complejidad técnica $TCF$ que se calcula con la ecuación \ref{eq:tcf}.

\begin{equation} \label{eq:tcf}
  TCF = 0,6 + 0,01 \times \Sigma (Peso_i \times ValorAsignado_i)
\end{equation}

\par Hay que calcular el factor de complejidad técnica, que se calcula mediante la cuantificación de un conjunto de factores, expuestos en la tabla \ref{tab:tcf}. El sumatorio $\Sigma (Peso_i \times ValorAsignado_i)$, donde el peso de cada factor ($Peso_i$) viene dado por el método de estimación y el valor asignado es ($ValorAsignado_i$) toma un valor del 0 al 5, dónde el 0 es que ese factor es irrelevante y el 5 significa que es muy relevante.

\bigskip
\par A continuación se justifican los valores asignados a cada factor en la tabla \ref{tab:tcf}:
\begin{itemize}[-]
  \item \textbf{Sistema distribuido:} El sistema posee un cierto nivel de distribución, ya que algunos de sus subsistemas se procesan en computadoras separadas por eso le asignamos un valor de 3.
  \item \textbf{Tiempo de respuesta:} Que el sistema responda rápidamente a las acciones de es fundamental, la velocidad de respuesta es un factor crítico para nuestra aplicación, por ello tiene un valor de 5.
  \item \textbf{Eficiencia por el usuario:} No es muy importante que los usuarios realicen las tareas eficientemente a través del software. Las tareas que tienen que realizar son muy sencillas, ya que es el software quién realiza la mayor parte de las acciones. Tiene un valor de 2.
  \item \textbf{Proceso interno complejo:} El proceso interno de la aplicación es muy complejo, tiene que aplicar algoritmos de \textit{machine learning} entre otras cosas, por ello recibe un valor de 5.
  \item \textbf{Re-usabilidad:} Es crítico que la aplicación pueda ser reutilizada en diferentes modelos de coches del cliente, por lo que este apartado recibe un valor de 5.
  \item \textbf{Facilidad Instalación:} No es fundamental que el sistema sea fácil de instalar, ya que no van a ser los usuarios finales los encargados de esta tarea, recibe un valor de 3.
  \item \textbf{Facilidad de uso:} Es importante que el sistema sea intuitivo y sencillo de utilizar porque se pone en riesgo la seguridad de las personas, por ello este apartado recibe un valor de 5.
  \item \textbf{Portabilidad:} El sistema está pensado para usarse siempre en sistemas similares. Recibe un valor de 1.
  \item \textbf{Facilidad de cambio:} Introducir actualizaciones es un valor añadido para el producto, por lo tanto que sea sencillo de cambiar es muy importante, entonces recibe un valor de 5.
  \item \textbf{Concurrencia:} La ejecución simultanea con varios hilos es importante para el softare a pesar de que se trata de un sistema distribuido. Recibe una valoración de 3.
  \item \textbf{Objetivos especiales de seguridad:} La seguridad es imprescindible en nuestro sistema ya que un fallo de seguridad podría provocar accidentes, por ello el valor de este apartado es de 5.
  \item \textbf{Acceso directo a terceras partes:} No es importante que sea accesible a terceras partes, por lo que recibe una valoración de 1.
  \item \textbf{Facilidades especiales de entrenamiento a usuarios finales:} No es algo importante para la aplicación, será algo de lo que tendrá que encargarse la empresa de automoción, incluyendo un manual de instrucciones junto al vehículo. Recibe un valor de 1.
\end{itemize}

\par El resultado de aplicar la ecuación \ref{eq:tcf} con el peso y el valor asignado a los factores ambientales es el mostrado en la ecuación \ref{eq:tcfresult},

\begin{equation} \label{eq:tcfresult}
  TCF = 0,6 + 0,01 \times \Sigma (Peso_i \times ValorAsignado_i) = 0,6 + 0,01 \times 120 = 1,04
\end{equation}

\begin{table}[h]
\begin{center}
\begin{tabular}{l l c c c}
\textbf{TFC} & \textbf{DESCRIPCIÓN} & \textbf{PESO} & \textbf{VALOR} & \textbf{FACTOR}\\ \hline \hline
T1	&	Sistema distribuido	&	2	&	3	&	6	\\
T2	&	Tiempo de respuesta	&	1	&	5	&	5	\\
T3	&	Eficiencia por el usuario	&	1	&	2	&	2	\\
T4	&	Proceso interno complejo	&	1	&	5	&	5	\\
T5	&	Re-usabilidad	&	1	&	5	&	5	\\
T6	&	Facilidad Instalación	&	0,5	&	3	&	1,5	\\
T7	&	Facilidad de uso	&	0,5	&	5	&	2,5	\\
T8	&	Portabilidad	&	2	&	1	&	2	\\
T9	&	Facilidad de cambio	&	1	&	5	&	5	\\
T10	&	Concurrencia	&	1	&	3	&	3	\\
T11	&	Objetivos especiales de seguridad	&	1	&	5	&	5	\\
T12	&	Acceso directo a terceras partes	&	1	&	1	&	1	\\
T13	&	Facilidades especiales de entrenamiento a usuarios finales	&	1	&	1	&	1	\\ \hline
\textbf{TOTAL} & & & & \textbf{120}\\ \hline \hline
\end{tabular}
\caption{Peso de los factores de complejidad técnica.}
\label{tab:tcf}
\end{center}
\end{table}

\par Por otro lado hay que calcular el peso de los factores ambientales que se calcula con la fórmula de la ecuación \ref{eq:ef}. Las variables $Peso_i$ y $ValorAsignado_i$ tienen el mismo significado y toman los mismos valores que en la ecuación \ref{eq:tcf}.

\begin{equation} \label{eq:ef}
  EF = 1,4 - 0,03 \times \Sigma (Peso_i \times ValorAsignado_i)
\end{equation}

\begin{table}[h]
\begin{center}
\begin{tabular}{l l c c c}
\textbf{EF} & \textbf{DESCRIPCIÓN} & \textbf{PESO} & \textbf{VALOR} & \textbf{FACTOR}\\ \hline \hline
E1	&	Familiaridad con el modelo del proyecto usado	&	1,5	&	2	&	3	\\
E2	&	Experiencia en la aplicación	&	0,5	&	2	&	1	\\
E3	&	Experiencia OO	&	1	&	4	&	4	\\
E4	&	Capacidad del analista líder	&	0,5	&	4	&	2	\\
E5	&	Motivación	&	1	&	5	&	5	\\
E6	&	Estabilidad de los requerimientos	&	2	&	5	&	10	\\
E7	&	Personal media jornada	&	-1	&	0	&	0	\\
E8	&	Dificultad en lenguaje de programación	&	-1	&	4	&	-4	\\ \hline
\textbf{TOTAL} & & & & \textbf{21}\\ \hline \hline
\end{tabular}
\caption{Peso de los factores ambientales.}
\label{tab:ef}
\end{center}
\end{table}

\bigskip
\par A continuación se justifican los valores asignados a cada factor en la tabla \ref{tab:ef}:
\begin{itemize}[-]
  \item \textbf{Familiaridad con el modelo de proyecto usado:} Los trabajadores están familiarizados con el modelo de proyecto que se va a utilizar.
  \item \textbf{Eperiencia en la aplicación:} Los trabajadores no poseen de experiencia en la aplicación, por lo que se asigna un valor de 2.
  \item \textbf{Experiencia OO:} La experiencia con la programación orientada a objetos es alta entre los miembros del equipo.
  \item \textbf{Capacidad del analísta líder:} El analísta líder es un trabajador con experiencia a los mandos de proyectos de software importantes, por lo tanto el valor en este apartado es de 4.
  \item \textbf{Motivación:} La plantilla se encuentra plenamente motivada, el valor es de 5.
  \item \textbf{Estabilidad de los requerimientos:} Aunque los requisitos se encuentran sujetos a cambios, las funcionalidades esenciales no van a ser modificadas. Valor de 5 en este apartado.
  \item \textbf{Personal a media jornada:} No contamos con trabajadores a media jornada, por lo que este apartado no computa.
  \item \textbf{Dificultad en el lenguaje de programación:} Utilizar un lenguaje de programación novedoso para los miembros del equipo de trabajo puede dificultar el proyecto. Recibe un valor de 4.
\end{itemize}

\par Una vez asignados los valores a los factores ambientales se realiza el cálculo propuesto en la ecuación \ref{eq:ef}.

\begin{equation} \label{eq:efresult}
  EF = 1,4 - 0,03 \times \Sigma (Peso_i \times ValorAsignado_i) = 1,4 - 0,03 \times 21 = 0,77
\end{equation}

\par Por último se cálcula el número de horas de programación aplicando la ecuación \ref{eq:puntoscasosuso} a los resultados de las ecuaciones \ref{eq:resultpuntoscasosusosinajustar}, \ref{eq:tcfresult} y \ref{eq:efresult}. El resultado de este cálculo se aprecia en la ecuación \ref{eq:resultpuntoscasosuso} y el resultado, al multiplicarlos por 20 ($105,70\times20=2.114,11$), da el número de horas hombre en la tarea de programación que requiere el proyecto. De este resultado, realizando un cálculo de proporciones, se obtienen los resultados en las diferentes tareas del proyecto como se muestra en la tabla \ref{tab:porcentajeAct}.

\begin{equation} \label{eq:resultpuntoscasosuso}
  UCP = UUCP \times TCF \times EF = 132 \times 1,04 \times 0,77 = 105,70
\end{equation}

\begin{table}[h]
\begin{center}
\begin{tabular}{l c c}
\textbf{ACTIVIDAD} & \textbf{PORCENTAJE} & \textbf{REAL}\\ \hline \hline
Análisis & 10\% & 528,52 \\
Diseño & 20\% & 1057,05 \\
Programación & 40\% & 2.114,11 \\
Pruebas & 15\% & 792,79 \\
Sobrecarga & 15\% & 792,79 \\ \hline
\textbf{TOTAL} & \textbf{100\%} & \textbf{5.285,28}\\ \hline \hline
\end{tabular}
\caption{Peso de los factores ambientales.}
\label{tab:porcentajeAct}
\end{center}
\end{table}

% \par Para la estimación del tiempo total que se ha de dedicar al proyecto se ha usado la estimación por casos de uso. De esta forma, lo primero que se ha realizado ha sido identificar los actores de que actuarán en los casos de uso. Estos actores son los objetos (árboles, postes y similares), el vehículo del usuario y el propio conductor, las señales de tráfico y el GPS. Así, y como se puede ver en la tabla \ref{tab:uaw}, se le asigna una dificultad y una ponderación a cada uno de los actores que participan en los casos de uso. Ello refleja el \textit{factor de peso de los actores sin ajustar}. Sin embargo, no basta con el factor de peso de los actores, sino que se ha de calcular el \textit{factor de peso de los casos de uso sin ajustar}. En este caso, se le han asignado los factores de ponderación en función del tipo de caso como puede verse en la tabla \ref{tab:ucw}. Así, multiplicando el número de casos de uso (\textit{\#CASOS DE USO}) por la ponderación asignada, calculamos el total de puntos de cada tipo de dificultad. De esta forma, el total de los \textit{puntos de casos de uso sin ajustar} sería 130, el equivalente a la suma de los totales de \ref{tab:uaw} y \ref{tab:ucw}.
%
% \par Tras ello, se han calculado los factores de dificultad técnica y los ambientales (véanse las tablas \ref{tab:tcf} y \ref{tab:ef}). Con ello, multiplicando los factores ambientales (1,4 - 21*0,03 = 0,77), los técnicos (0,6 + 44*0,01 = 1,04) y el total de puntos de casos de uso sin ajustar, se ha obtenido que el total de puntos de casos de uso ajustados son 105,70. De esta forma, se ha calculado que el total de horas hombre es de 2114,11 en tiempo de programación. Por ello, se ha calculado que el total de horas que se deberán dedicar a este proyecto es de $2114,11/0,4=5285,28$, ya que, como se puede ver en la tabla \ref{tab:porcentajeAct}, el porcentaje que se estima se dedicará a programación será del 40\%.
% \par Con esta estimación de horas totales, en la tabla \ref{tab:repHoras} (Documento de Costes), como ya se ha mencionado, se ha realizado el reparto de horas entre los miembros del equipo de forma que el total de las mismas coincida con el total estimado.
