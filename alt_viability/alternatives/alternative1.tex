\subsubsection{Alternativa 1: SharePoint}

\begin{center}
\begin{longtable}{p{4cm} p{8cm}}

%HEAD
\textbf{ASPECTO} & \textbf{VALORACIÓN} \\ \hline \hline
\endfirsthead
\endhead

%FOOT
\multicolumn{2}{r}{\textit{Continúa en la siguiente página}} \\
\endfoot
\endlastfoot

%table
\textbf{Tecnología} & SharePoint (Microsoft) está escrito en C#\\ \hline

\textbf{Dificultad de aprendizaje de la tecnología} & Media\\ \hline

\textbf{Coste de la licencia de la tecnología} &
\par El coste de uso de SharePoint es de 8,40 \euro por usuario al mes. Ello incluye todas las características y funcionalidades para la empresa, aun que se trata de una solución on-line.
\par Por otro lado, la solución Office 365 Enterprise E3, que incluye SharePoint Enterprise, tiene un coste de 19,70 \euro por usuario y mes, y van más dirigida a grandes empresas.
\\ \hline

\textbf{Tiempo requerido para elaborar la solución} & Dado de que SharePoint ofrece soluciones CRM con acceso a distintas plantillas y recursos, el tiempo requerido para nuestro proyecto rondaría los 4 meses, no siendo este muy alto y dedicando parte del mismo al aprendizaje del uso de la tecnología.\\ \hline

\textbf{Requisitos SW/HW} & Mínimo de 4 a 8 GB de RAM y 80 GB de almacenamiento y base de datos integrada.\\ \hline

\textbf{Extensión en el mercado} & Su uso en el marcado está bastante extendido. No obstante, al tratarse de una solución de pago, muchas de las pequeñas y medianas empresas recurren a soluciones OpenSource. Actualmente, las grandes empresas también están apostando por tecnologías más avanzadas como Liferay.\\ \hline \hline


\caption{Alternativa 1: SharePoint}\\
\label{tab:alternative1}
\end{longtable}
\end{center}
