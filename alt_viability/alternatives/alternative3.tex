\subsubsection{Alternativa 3: Liferay Portal}

\begin{center}
\begin{longtable}{p{4cm} p{8cm}}

%HEAD
\textbf{ASPECTO} & \textbf{VALORACIÓN} \\ \hline \hline
\endfirsthead
\endhead

%FOOT
\multicolumn{2}{r}{\textit{Continúa en la siguiente página}} \\
\endfoot
\endlastfoot

%table
\textbf{Tecnología} & Liferay está escrita en Java, tecnología muy madura y ampliamente usada.\\ \hline

\textbf{Dificultad de aprendizaje de la tecnología} & Media\\ \hline

\textbf{Coste de la licencia de la tecnología} &
\par La versión comunitaria (CommunityEdition) es gratuita. La versión más recomendada para grandes empresas (EnterpriseEdition) dado que ofrece servicio técnico y soluciones personalizadas tiene un coste alto, dependiente del proyecto concreto.
\\ \hline

\textbf{Tiempo requerido para elaborar la solución} & Dado de que Liferay ofrece soluciones CRM con acceso a distintas plantillas y recursos, el tiempo requerido para nuestro proyecto rondaría los 4 meses, no siendo este muy alto y dedicando parte del mismo al aprendizaje del uso de la tecnología.\\ \hline

\textbf{Requisitos SW/HW} & Mínimo de 4 GB de RAM y 40 GB de almacenamiento para entorno de producción (al menos 4 CPU).\\ \hline

\textbf{Extensión en el mercado} & Muchos de los portales corporativos de las grandes empresas están realizados con Liferay, como por ejemplo Inditex, EducaMadrid, UCM o Vodafone.\\ \hline \hline


\caption{Alternativa 3: Liferay}\\
\label{tab:alternative3}
\end{longtable}
\end{center}
