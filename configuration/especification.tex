\subsection{Especificación de requisitos}
\par En este apartado se identifican las tareas de coordinación y gestión que serán necesarias para llevar a cabo la
SCM. El SCMR será el encargado de realizar las siguientes actividades: definir ítems de configuración, definir un
ambiente para llevar el control de cambios sobre estos ítems, definir el proceso de cambios, mantener la línea
base del proyecto, controlar cambio importantes sobre la línea base del proyecto y auditar la estabilidad de la
línea base.
\par Este apartado está compuesto por tres puntos: la organización del proyecto, las responsabilidades del SCRM,
y las referencias a las políticas que se van a aplicar en este proyecto.

\subsubsection{Organización}
\par Debe existir contacto permanente y directo entre el personal desarrollador y el comité de control de cambios,
de modo que las demoras en la tramitación de un cambio sean lo más cortas posible, de modo que los procesos
tanto de mejora como de corrección no sean un trabajo tedioso. Tanto el comité de control de cambios como el
resto de personal desarrollador deben prestar especial atención a los puntos en los que se ha estipulado que se
van a establecer líneas base dentro del desarrollo.

\subsubsection{Responsabilidades}
\begin{description}[style=multiline, leftmargin=4cm]
  \item[Comité de control de cambios:] conjunto de personas encargadas de valorar las ventajas y los inconvenientes de las solicitudes de cambios que pueden afectar al proyecto, de tal manera que el impacto que puede producir dichos cambios sea mínimo.  Estas personas deben evaluar las peticiones de cambio, aceptándolas o rechazándolas.
  \item[Responsable de GC:] es la persona encargada de la planificación de la configuración. Además es su responsabilidad definir las líneas base y asegurar su seguimiento. Como actividades secundarias, también participa en la implantación del producto con el cliente y es el responsable del control de cambios, es decir, debe reportar los cambios no autorizados e identificar y controlar los cambios en los CI. Es también el encargado de aprobar los cambios estructurales en la base de datos de configuración.
  \item[Bibliotecario:] es el encargado de establecer y mantener el software y la documentación de cada proyecto de acuerdo a un proceso documentado. Debe proveer a los desarrolladores copias sobre las líneas base del proyecto e informarles sobre los cambios que surjan en relación a los elementos de configuración.
  \item[Resto personal desarrollador:] este equipo debe revisar y realizar observaciones sobre el SCM, ya que posteriormente deberá de implementar las actividades de acuerdo al plan. Deberán participar también en la solución a los problemas del SCM que sean de su competencia. Por último, deberán implementar las prácticas, procesos y procedimientos definidos en el plan del proyecto y en otros planes o documentos complementarios.
\end{description}

\begin{table}[h]
\begin{center}
\begin{tabular}{ p{5cm} | r }
\hline
	\textbf{Responsable} & \textbf{Ocupación} \\ \hline \hline
  JorgeHeviaJP & SCMR \\
  CarlosOlivaresCF & GC \\
  CarlosOlivaresAS & Bibliotecario \\
  JorgeHeviaDJ, CarlosOlivaresD1, LuisCabreroD2 & Equipo de desarrollo \\ \hline
\end{tabular}
\caption{Responsabilidades.}
\label{tab:Responsabilidades}
\end{center}
\end{table}


\subsubsection{Políticas, directivas y procedimientos aplicables}
\par Durante el proceso de documentación se y desarrollo de todo el proyecto se van a utilizar la herramienta GIT, almacenando el proyecto en un repositorio privado de la empresa BitBucket, permitiendo un control de versiones muy eficiente. Toda la documentación será llevada a cabo en el lenguaje LaTex, por lo que utilizaremos también GIT para su control de versiones. Sin embargo, con el fin de facilitar el desarrollo de la documentación, se realizarán en Google Docs varias versiones, para posteriormente pasarlas a LaTex.

\begin{itemize}[-]
  \item \textbf{Políticas de configuración de código fuente y documentación de usuario:} se utilizará un repositorio privado en la empresa BitBucket, que implementa un sistema GIT, para llevar a cabo un control de las versiones. Cada \textit{commit} será explicativo, lo que ayudará a todo el equipo.
  \item \textbf{Política de almacenamiento:} se utilizará Google Drive junto con Google Docs para las primeras versiones de la documentación. Una vez el contenido del documento sea firme, será transcrito a LaTex y almacenado junto con el código fuente en el repositorio de BitBucket.
  \item \textbf{Políticas de cambios:} los documentos únicamente podrán ser modificados por el responsable de la gestión de configuración (CCR) y solo cuando el Comité de Control de Cambios lo estime oportuno. Cualquier miembro del equipo podrá proponer una solicitud para cambiar o revisar cualquier parte del proyecto. Esta solicitud será comunicada al CCR y posteriormente se remitirá al CCC.
  \item \textbf{Política de confidencialidad:} todo los documentos relacionados con el proyecto, ya sean para uso interno del equipo o para el cliente tendrán un carácter confidencial.
\end{itemize}
